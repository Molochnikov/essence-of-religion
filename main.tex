\documentclass[12pt]{article}
\usepackage[utf8]{inputenc}
\usepackage[a6paper]{geometry}
\usepackage[T2A]{fontenc}
\usepackage[russian]{babel}
\usepackage{indentfirst}
\usepackage{hyperref}
\usepackage{titlesec}

\title{Сущность Религии}
\author{Людвиг Фейербах}
\date{1846 г.}

\tolerance=10000
\hbadness=10000
\vbadness=10000

\begin{document}

\maketitle

\tableofcontents

\section{}

Эта работа представляет собой ту <<статью>>, на которую я указал в <<Лютере>>, но она дается не в форме статьи, а в виде ряда независимых самостоятельных мыслей. Предмет этих мыслей, или во всяком случае отправная точка их, сводится к религии, поскольку объектом религии является природа; в <<Сущности христианства>> и в <<Лютере>> я отвлекся от природы, да и должен был отвлечься по самому замыслу работы: ведь специфическая сторона христианства заключается не в боге, раскрывающемся в природе, а в боге, данном в человеке.

Существо в его отличии и независимости от человеческой сущности или бога, как он истолковывается в <<Сущности христианства>>, существо без человеческой сущности, без человеческих свойств и человеческой индивидуальности в действительности есть не что иное, как природа. Для меня природа, так же как и <<дух>>, есть не что иное, как общий термин для обозначения существ, вещей и предметов, отличаемых человеком от самого себя и от своего творчества и объединенных под общим названием природы, но это не есть всеобщая сущность, отвлеченная и отмежеванная от действительных вещей, не есть некая персонификация и мистификация.



\section{}

Основу религии составляет чувство зависимости человека; в первоначальном смысле природа и есть предмет этого чувства зависимости, то, от чего человек зависит и чувствует себя зависимым. Природа есть первый, изначальный объект религии, как это вполне доказывается историей всех религий и народов.



\section{}

Утверждение, что религия врождена человеку, что она есть нечто естественное, --- ложно, если религию в ее общем смысле подменять идеями теизма, то есть верой в бога в собственном смысле; но это утверждение совершенно справедливо, если под религией понимать не что иное, как чувство зависимости, --- чувство или сознание человека, что он не существует и не может существовать без другого, отличного от него существа, что он своим существованием обязан не самому себе. В этом смысле религия так же близка человеку, как свет глазу, как воздух легким, как пища желудку. Религия есть восчувствование и признание того, чем я являюсь. Но прежде всего я не нечто, существующее без света, без воздуха, без воды, без земли, без пищи, я --- зависимое от природы существо. Для животного и зверообразного человека эта зависимость лишь бессознательная, непродуманная; возвыситься до религии --- значит довести эту зависимость до сознания, представить ее, почувствовать и признать. Таким образом, всякая жизнь зависит от смены времен года, но только человек отмечает эту смену драматическими образами, праздничными действиями. А такие празднества, выражающие и изображающие лишь смену времен года или фазы луны, --- древнейшее, первое религиозное исповедание человечества, акты веры в собственном смысле слова.



\section{}

Определенный человек, этот народ, это племя зависят не от природы в общем смысле слова, не от земли вообще, но от этой почвы, от этой страны, он находится в зависимости не от воды вообще, а от этой воды, от этого потока, от этого источника. Египтянин вне Египта не египтянин, индиец вне Индии --- не индиец. Поэтому с полным правом, с тем самым правом, с которым универсальный человек почитает свою универсальную сущность, как бога, древние, ограниченные народы, привязанные телом и душой к своей почве, усматривавшие свою сущность не в своей человечности, а в своих народных и племенных особенностях, молились горам, деревьям, животным, рекам и источникам своей страны, как божественным существам; ведь все их бытие, вся их сущность всецело коренились в особенностях их страны, в особенностях их природы.



\section{}

Это совершенно фантастическое представление, будто человек смог возвыситься над своим животным состоянием только благодаря провидению, содействию <<сверхчеловеческих>> существ, богов, духов, гениев, ангелов. Разумеется, человек стал тем, что он есть, не самостоятельно и не исключительно только через самого себя; ему была нужна для этого поддержка других существ. Но эти существа не были сверхприродными, воображаемыми созданиями, но были действительными, естественными существами; не были существами, стоящими над человеком, а были ему подчинены; в самом деле, вообще все то, что поддерживает человека в его сознательной и произвольной деятельности (ведь обычно такие дела только и называются человеческими делами), всякий благой дар и природный задаток ниспосылаются не свыше, а возникают снизу, сваливаются не с высот, а порождаются из глубин природы. Такими доставляющими помощь существами, такими гениями-хранителями человека были по преимуществу животные. Только благодаря животным человек возвысился над животным царством, только благодаря их охране и содействию мог взойти посев человеческой культуры. В Зенд-Авесте, а именно в Вендидаде, как известно, самой древней и подлинной части Зенд-Авесты, читаем: <<мир существует благодаря уму собаки. Хотя эта часть ,,составлена лишь в позднейшее время``. Если бы собака не охраняла улиц, то разбойники и волки расхитила бы все имущество>>. Религиозное почитание животных вполне оправдывается в связи с их ролью для человека именно в эпоху зарождающейся культуры. Для человека животные были незаменимыми, необходимыми существами; его человеческое существование зависело от них; а то, от чего зависит жизнь, существование человека, --- для него бог. Если христиане больше не почитают природы, как бога, то это происходит только потому, что согласно их религиозным представлениям их существование зависит не от природы, но от воли существа, отличного от природы; вместе с тем они рассматривают и почитают это существо как существо божественное, то есть высшее, только потому, что они признают его за виновника и хранителя их бытия, их жизни. Таким образом почитание бога зависит только от почитания человеком самого себя, богопочитание есть проявление такого самопочитания. Если я презрительно отношусь к самому себе и к своей жизни, как я мог бы высоко расценивать и почитать то, от чего зависит эта злосчастная, презренная жизнь, --- нужно при этом учитывать, что по первоначальному, естественному представлению человек не отличает себя от своей жизни. Придавая ценность источнику жизни, я лишь в предмете своего сознания начинаю усматривать ту ценность, которую бессознательно придаю себе, своей жизни. Поэтому, чем ценнее оказывается жизнь, тем, естественно, более высокую ценность и достоинство приобретают податели жизненных благ --- боги. Как могли бы боги блистать в золоте и серебре, раз человек еще не знает цены и употребления серебра и золота? Различие между полнотой и жизнерадостностью бытия греков и отвращением и презрением к жизни индейцев весьма значительно; но также весьма значительна разница между греческой мифологией и моралью индейских басен, между олимпийским отцом богов и людей и великой индейской сумчатой крысой или гремучей змеей --- прародительницей индейцев!



\section{}

Христиане, подобно язычникам, радуются жизни, но свои благодарственные молитвы за блага жизни они возносят к небесному отцу; именно потому они упрекают язычников в идолопоклонстве, что те в своей благодарности, в своем культе ограничиваются тварями и не возвышаются до первопричины, единственно подлинной причины всех благодеяний. Но ведь не Адаму же, первому человеку, я обязан своим существованием? Почитаю ли я его за своего отца? Почему мне не остановиться на твари? Разве сам я --- не тварь? Для меня самого, не издалека пришедшего, для меня, как этого определенного индивидуального существа, не является ли эта ближайшая, эта также определенная причина последней причиной? Разве эта моя индивидуальность, неотторжимая, неотличимая от меня самого и моего существования, не находится в зависимости от индивидуальности моих родителей? Восходя все дальше, не теряю ли я в конце концов всякий след моего существования? Разве для этого попятного хода нет неизбежной остановки и границы? Разве первоисточник моего бытия не абсолютно индивидуальный? Разве я рожден и зачат в том самом году, в тот самый час, в том же расположении, одним словом, при тех же внутренних и внешних условиях, как и мой брат? Итак, мое порождение не так же ли своеобразно и индивидуально, как безусловно индивидуальна моя жизнь? Должен ли я свое почитание простирать до Адама? Нет! Я с полным правом останавливаюсь в своем религиозном почитании на ближайших ко мне существах, на моих действительных родителях, как на виновниках моего существования.



\section{}

Непрерывный ряд так называемых конечных причин или вещей, определявшийся прежними атеистами как нечто бесконечное, а теистами --- как нечто конечное, существует лишь в мысли, в человеческих понятиях, подобно времени, где неотступно и неизменно каждое мгновение присоединяется к предшествующему. В действительности однообразное безразличие этого причинного ряда прерывается, упраздняется различием, индивидуальным характером вещей, представляющим нечто новое, самостоятельное, единственное, окончательное, абсолютное. Разумеется, священная вода по смыслу естественной религии есть нечто сложное, зависящее от водорода и кислорода, но вместе с тем это --- новая, самодовлеющая, своеобразная сущность, в которой свойства обоих веществ, как таковые, исчезают, упраздняются. Разумеется, лунный свет, почитаемый язычником с его простодушным вероучением, как самостоятельный свет, есть свет заимствованный, но вместе с тем он отличается от непосредственного света солнца, он есть самобытный свет, модифицированный сопротивляемостью луны; итак, это свет, которого бы не было при отсутствии луны, своеобразие этого света определяется только ею. Разумеется, собака, которую перс за ее бдительность, готовность услужить и верность призывает в своих молитвах как благодетельное и поэтому божественное --- существо, есть творение природы, которое не само по себе есть то, что оно есть; и вместе с тем только сама собака, именно это и никакое другое существо, обладает такими достойными почитания свойствами. Должен ли я в связи с этими свойствами возносить свои очи к универсальной первопричине и повернуться спиной к собаке? Но ведь всеобщая причина одинаково оказывается как причиной дружественно настроенной к человеку собаки, так и враждебного человеку волка, а ведь если я хочу утвердить свое собственное, более ценное бытие, то вразрез со всеобщей причиной я должен стремиться к уничтожению волка.



\section{}

Божественная сущность, раскрывающаяся в природе, есть не что иное, как сама природа; она раскрывается, выявляется и напрашивается человеку как божественное существо. У древних мексиканцев среди многих богов существовал также бог соли. Скорее это богиня, но в данном случае это безразлично. Этот бог соли убедительным образом открывает нам тайну божественности природы вообще. Соль (каменная соль) олицетворяет для нас экономические, медицинские и технологические действия природы в ее столь восславленной теистами полезности и благодетельности; соль своим воздействием на глаза и настроение, своим цветом, своим блеском, своей прозрачностью олицетворяет красоту природы; своей кристаллической структурой и формой олицетворяет гармонию и закономерность природы; своим составом из противоположных веществ --- связь противоположных элементов природы в виде одного целого, связь, в которой теисты с давних пор усматривали неопровержимое доказательство существования царя природы, отличного от нее самой, так как благодаря незнанию природы они не понимали, что как раз взаимным притяжением обладают противоположные вещества и существа, сами собой соединяющиеся в единое целое. Что же такое представляет собой бог соли? Тот бог, чье царство, бытие, откровение, действие и свойства содержатся в соли? Это не что иное, как сама соль, по своим свойствам и действиям раскрывающаяся человеку как божественное, то есть благодетельное, величественное, ценнейшее и удивительное существо. Гомер определенно называет соль божественной; таким образом, бог соли есть только знак и выражение божества, или божественной соли, точно так же бог вселенной, или вообще природы, есть только знак и выражение божественности природы.



\section{}

Вера в то, что в природе раскрывается другое существо, а не сама природа, что природа восполняется и управляется существом, от нее отличным, принципиально совпадает с верой в то, что духи, демоны, дьявол раскрываются в человеке, во всяком случае в некоторых состояниях, что они владеют человеком; это фактически есть вера в то, что природа одержима чуждым, призрачным существом. Разумеется, и в действительности природа с точки зрения данного религиозного взгляда находится во власти духа, но этот дух есть дух человеческий, есть его фантазия, его чувство, которое бессознательно внедряется в природу, делает природу символом и зеркалом своей сущности.



\section{}

Природа --- не только первый, основной предмет религии, она также неизменно наличная основа, постоянный, хотя и скрытый, фон религии. Вера в то, что бог, если даже он мыслится как отличное от природы, сверхприродное существо, есть объективная сущность, существующая помимо человека, по выражению философов, вера эта коренится лишь в том, что существующее помимо человека объективное существо --- вселенная, природа, прежде всего сама есть бог. Существование природы не опирается согласно взглядам теизма на существование бога. Нет! Наоборот: существование бога или, вернее, вера в его существование опирается лишь на существующую природу. Ты только потому принужден мыслить бога как реальное существо, что ты понуждаешься самой природой предпослать своему существованию и своему сознанию бытие природы; основное понятие бога сводится к тому, что он есть бытие, предшествующее твоему существованию. Другими словами: в вере, что бог существует помимо человеческого сердца и разума, что он существует безусловно, вне зависимости от того, существует ли человек или нет, мыслит ли человек бога или не мыслит, хочет ли он его или не хочет, --- в этой вере, или, скорее, в предмете этой веры, никакая другая сущность не маячит в твоей голове, кроме природы, бытие которой не опирается на существование человека, не говоря уже об обосновании со стороны человеческого ума и сердца. Поэтому если теологи, в особенности рационалисты, главное достоинство бога усматривают в том, что он есть существо, независимое от человеческой мысли, то пусть они обратят внимание на то, что честь такого существования принадлежит также богам слепых язычников, звездам, камням, деревьям и животным, что, таким образом, непричастное мысли существование их бога не отличается от существования египетского бога Аписа.



\section{}

Свойства, обусловливающие и выражающие отличие божественного существа от существа человеческого, или во всяком случае от человеческого индивидуума, прежде всего или в основном являются лишь свойствами природы. Бог есть могущественнейшее, или, вернее, всемогущее, существо: другими словами, он может то, чего не может человек, что, скорее, бесконечно превышает человеческие силы и поэтому внушает человеку смиренное чувство его ограниченности, бессилия и ничтожества. Бог говорит Иову: <<Можешь ли ты связать воедино семизвездие? Можешь ли ты развязать узы Ориона? Можешь ли ты посылать молнии, чтобы они пошли и сказали: вот где мы? Можешь ли ты коню дать силы? Твоею ли мудростью летает ястреб? Обладаешь ли ты мышцами бога и можешь ли ты греметь тем же голосом, что и он?>> Нет! Этого человек не может; человеческий голос нельзя сравнить с громом. Но что это за мощь, которая обнаруживается в силе грома, в крепости коня, в полете ястреба, в неудержимом течении семизвездия? Это мощь природы! Сократ не признавал физики, считая ее занятием, превосходящим человеческие силы и бесполезным; в самом деле, ведь если бы мы даже знали, как, например, образуется дождь, это не дало бы возможности вызывать дождь; поэтому оставалось только заниматься человеческими, моральными вопросами, доступными знанию. Это значит: доступное человеку есть нечто человеческое; чего человек не может сделать, есть нечто сверхчеловеческое, божественное. Так и один кафрский король говорил, что кафры <<\dotsверят в невидимую силу, доставляющую им то благо, то зло, посылающую ветер, гром и молнию и производящую все, чему они не могут подражать>>. Так, один индеец говорил миссионеру: <<Можешь ли ты сделать так, чтобы трава росла? Я не думаю, и никто этого не может, кроме великого Манито>>. Итак, основное понятие бога как существа, отличного от человека, есть не что иное, как природа. Бог есть существо вечное, но в самой Библии сказано: <<Поколение следует за поколением, земля же вечна>>. В Зенд-Авесте солнце и луна в связи с их неуничтожаемостью выразительно названы <<бессмертными>>. И один перуанский инка сказал доминиканцу: <<Ты клянчишь у бога, умершего на кресте, я же поклоняюсь неумирающему солнцу>>. Бог есть всеблагое существо, <<ибо он повелевает своему солнцу восходить над злыми и добрыми и посылает дождь на праведных и неправедных>>; но существо, не различающее между добром и злом, между праведным и неправедным, распределяющее блага жизни не по моральным заслугам, вообще потому производящее на человека впечатление благого существа, что его действия, например животворящий солнечный свет и дождь, являются источниками благотворнейших ощущений, --- существо это и есть природа. Бог есть всеобъемлющее, универсальное, самотождественное существо.

Но ведь то же самое солнце светит всем людям и обитателям земли или вселенной, поскольку первоначально и для всех религий земля и есть сам мир; ведь ото же самое небо нас всех обнимает, ведь та же самая земля нас всех на себе держит. Амвросий говорит: вся природа свидетельствует о бытии единого бога, ибо существует лишь один мир. У всех одно солнце, одна луна, одно небо, одна земля и одно море, говорит Плутарх. У одного они называются так, у другого --- иначе. Так, у вселенной один Дух-руководитель, но у него разные имена, и культы его различны. Бог <<не есть существо, пребывающее в храмах, созданных человеческими руками>>, но и природа не такое существо. Кто может заключить свет, небо, море в ограниченные человеческие пределы? Древние персы и германцы поклонялись только природе, у них не было храмов. Почитателю природы слишком тесно, слишком душно в искусственных, ограниченных постройках храма или церкви; он чувствует себя хорошо лишь под открытым, необъятным небом чувственного созерцания. Бог неопределим человеческим мерилом, он --- необъятное, великое, бесконечное существо. Но он таков только потому, что вселенная, им созданная, обширна, неизмерима, бесконечна или во всяком случае кажется таковой человеку. Произведение воздает хвалу своему мастеру: величие творца коренится лишь в величии творения. <<Как величественно солнце, но как велик создавший солнце!>>. Бог есть сверхземное, сверхчеловеческое, величайшее существо; но по своему происхождению и основанию он есть не что иное, как высшее существо в пространственном и оптическом отношении, а именно небо с его яркими явлениями. Все религии, если у них есть хоть какой-нибудь размах, переносят своих богов в сферу облаков, эфира или солнца, луны или звезд, в конце концов все боги теряются в синеве небес. Даже духовное божество христиан имеет свое пребывание, свое седалище наверху, в небе. Бог есть таинственное, непонятное существо, но только потому, что природа для человека --- а именно человека религиозного --- таинственна, непонятна. Бог говорит Иову: <<Знаешь ли ты, как расходятся облака? Бывал ли ты на дне морском? Знаешь ли ты размеры земли? Видел ли ты, откуда падает град?>> Наконец, бог есть существо, возвышающееся над человеческим произволом, непричастное человеческим потребностям и страстям, существо, себе тождественное, царящее по неизменным законам, непреклонно на все века утверждающее то, что оно раз установило. Но и это существо --- что оно такое, как не природа, неизменно пребывающая при всех сменах, закономерная, неумолимая, ни с чем не считающаяся и стихийная? Все эти свойства, первоначально взятые только из созерцания природы, затем превращаются в абстрактные, метафизические свойства, так же как природа становится абстрактной, мысленной сущностью. При этом взгляде, когда человек забывает происхождение бога из природы, когда бог уже не есть созерцаемое, чувственное существо, но существо мысленное, мы приходим к следующему выводу: отличный от подлинно человеческого бога, не антропоморфный бог есть не что иное, как сущность разума. Этого указания достаточно, чтобы пояснить отношение этой работы к моим сочинениям <<Лютер>> и <<Сущность христианства>>. Для понимающего этого довольно.



\section{}

Бога как творца природы мы себе представляем в виде существа, от природы отличного, но то, что охватывает и выражает это существо, его действительное содержание, есть только природа. <<По плодам их узнаете их>>, --- читаем мы в Библии; также. Павел выразительно указывает нам на вселенную, как на то творение, из которого следует постигнуть бытие и существо божие, ибо то, что сотворено кем-нибудь, содержит его сущность, показывает нам, кто он и что он может. Поэтому то, что мы имеем в природе, мы имеем в боге, взятом в смысле творца, или причины природы, --- следовательно, это не моральное, не духовное, но только естественное, физическое существо. Чистым культом природы оказался бы такой культ, который опирается на бога только как творца природы, не связывая с ним никаких определений, почерпнутых из человека, и вместе с тем не представляя его в виде политического или нравственного, то есть человеческого законодателя. Правда, творцу природы мы приписываем ум и волю; но то, чего хочет эта воля, то, что мыслит этот ум, есть как раз то, для чего не нужно ни воли, ни ума, для чего достаточны обыкновенные механические, физические, химические, растительные, животные силы и импульсы.



\section{}

Образование ребенка во чреве матери, биение сердца, пищеварение и другие органические функции не являются действиями ума и воли, так же точно природа вообще не есть действие духовной сущности, то есть наделенной волей или мыслью. Если с самого начала природа есть духовный продукт и, следовательно, есть проявление духа, то и наличные явления природы суть духовные действия, явления духа. Кто говорит А, тот должен сказать Б; сверхъестественное начало неизбежно требует сверхъестественного продолжения. Ведь только там человек считает волю и ум причиной природы, где действие, стоящее ниже воли и рассудка, господствует над человеческим умом, где он все объясняет только из себя, только человеческими мотивами, где он ничего не понимает и не знает об естественных причинах, где он поэтому и отдельные наличные явления природы выводит из бога или из подчиненных духов, --- так он, например, объясняет непонятные движения звезд. Но если теперь опора земли и созвездий не сводится к всемогущему слову божию, если источник их движения не духовный или ангельский, а механический, то неизбежно и причина, а именно первопричина этого движения, оказывается механической или вообще естественной. Выводить природу из воли и ума, вообще из духа, --- значит путать все счета, это значит не знающей мужа деве давать родить спасителя только духом святым, это значит превращать воду в вино, это значит словами заклинать бури, это значит словами двигать горы, это значит словами делать слепых зрячими. Как это беспомощно, мелко отвергать подчиненные причины, вторичные причины, суеверия, --- чудеса, дьявола и духов, --- как основание для объяснения естественных явлений, а первопричину всяческого суеверия оставлять неприкосновенной.



\section{}

Многие отцы церкви утверждали, что сын божий не есть действие воли, но сущности, природы бога, что естественный продукт предшествует волевому действию и поэтому акт рождения как сущностный или естественный акт предшествует акту творения как акту воли. Так заявила свои права истина природы даже в пределах сверхъестественного бога, хотя это и было в величайшем противоречии с его сущностью и волей. Акт рождения предваряет акт воли, деятельность природы первоначалънее деятельности сознания, деятельности воли. Совершенно верно. Сначала должна быть природа, прежде чем появится то, что от природы отличается, что противопоставляет себе природу как предмет хотения или мысли. Идти от отсутствия мысли к уму --- это путь житейской мудрости, идти же от ума к отсутствию ума --- это прямой путь в сумасшедший дом теологии. Не давать духу опоры в природе и, наоборот, сводить природу к духу --- это значит не голову ставить над брюхом, а брюхо над головой. Высшее предполагает низшее, а не последнее --- первое по той простой причине, что высшее должно иметь нечто под собой, чтобы стоять выше. Логически возможно, что низшее предполагает высшее, но это никогда не бывает реально-генетически. И, чем выше, чем значительнее существо, тем больше условий оно предполагает. Поэтому не первое существо, но позднейшее, последнее, самое зависимое, самое нуждающееся и самое сложное существо есть существо величайшее, так же как и в истории земли самыми тяжелыми и плотными оказываются не древнейшие, первые породы, сланцевые и гранитные, но позднейшие, новейшие, базальты и плотные лавы. Существо, которому принадлежит честь ничего не предполагать для себя, обладает также честью быть ничем. Но надо признаться, христиане владеют искусством делать из ничего нечто.



\section{}

Все вещи происходят и зависят от бога, так говорят христиане согласно своей блаженной вере, но, добавляют они тотчас согласно своему безбожному уму, это происходит только опосредствованным путем: бог есть лишь первопричина, после него выступает на сцену необозримая толпа подчиненных богов, полчища посредствующих причин. Но так называемые посредствующие причины --- единственно действительные и действенные, единственно предметные и осязательные причины. Бог, который больше не побивает людей стрелами Аполлона, который больше не потрясает душ молнией и громом Юпитера, который кометами и другими огненными явлениями не разжигает больше ада для закоренелых грешников, который высочайшей <<всемогущей>> рукой не притягивает железа к магниту, не вызывает отливов и приливов и не защищает суши от своевольных вод, всегда грозящих новым потопом, --- словом, бог, изгнанный из царства опосредствующих причин, есть только причина по имени, безвредное, очень скромное, мысленное существо, --- простая гипотеза для разрешения теоретической трудности, для объяснения первоначального возникновения природы, или, вернее, органической жизни. В самом деле, допущение существа, отличного от природы, для ее объяснения опирается, по крайней мере в последней инстанции, только на необъяснимость возникновения органической, в особенности человеческой, жизни из природы, --- впрочем, эта необъяснимость лишь относительная, субъективная; при этом теист превращает свою неспособность объяснить жизнь из природы в неспособность природы породить жизнь из себя самой; таким образом, он превращает границы своего ума в пределы природы.



\section{}

Творчество и сохранение неразрывно друг с другом связаны. Поэтому если нашим творцом является существо, отличное от природы, --- бог, то он также наш хранитель; таким образом, нас охраняет не сила воздуха, тепла, воды, хлеба, но божественная сила. <<В нем мы живем, движемся и есмы>>. Лютер говорит: <<Не хлеб, а слово божие естественно питает и наше тело, как оно созидает и сохраняет все вещи; (,,Евреям`` 1)>>. <<Поскольку хлеб существует, то им а через него он (бог) насыщает, так чтобы мы не видели и не думали, что это делает хлеб; а где хлеба нет, там он питает без хлеба, только словом, как он в других случаях это делает под видом хлеба>>. <<Итак, все твари --- личины и маски бога, он им предоставляет действовать вместе с ним и помогает делать то, что он, впрочем, мог бы делать, да и делает, без их содействия>>. Если же нашей хранительницей оказывается не природа, а бог, то природа --- просто игра в прятки божества и, следовательно, лишняя, мнимая сущность, как и, наоборот, бог есть лишнее, мнимое существо, если мы хранимы природой. Но очевидно и бесспорно, что мы обязаны своим сохранением лишь особым действиям, свойствам и силам естественных существ; поэтому в конце концов мы не только имеем право заключить, что мы своим существованием обязаны только природе, но мы даже принуждены признать это. Мы живем среди природы, --- так неужели наше начало, наше происхождение находится вне природы? Мы живем в природе, с природой, на счет природы, --- так неужели мы произошли не от нее? Какое противоречие!

\section{}

Земля не всегда была такой, какова она в настоящее время; скорее она достигла своего теперешнего состояния в результате развития и ряда революций. Теперь мы благодаря геологии знаем, что на разнообразных ступенях развития существовали различные растения и животные, которых уже нет теперь или которые перестали существовать в один из предшествующих периодов. Впрочем, я не могу согласиться со взглядом, будто органическая жизнь шла путем строгой постепенности, что в определенное время существовали только улитки, моллюски и другие низшие формы, только рыбы, только амфибии. Этот взгляд приложим лишь к периоду серо-вакковой формации, если только подтвердилось открытие костей и зубов сухопутных млекопитающих в период каменноугольной формации. Так, теперь больше нет ни трилобитов, ни энкрипитов, ни аммонитов, ни птеродактилей, нет ихтиозавров, плезиозавров, нет мегатериев и динотериев и т. д. Почему же? Очевидно, потому, что нет соответствующих условий для их существования. Если же жизнь кончается вместе с исчезновением необходимых условий, то и начало этой жизни совпадает с возникновением этих условий. Даже и теперь, когда растения и животные, во всяком случае, несомненно, высшие, возникают лишь путем органического зарождения, мы видим, что, как только даны их особые жизненные условия, удивительным, еще необъяснимым путем немедленно в необозримом количество появляются также эти растения и животные. Поэтому, естественно, нельзя себе представить возникновение органической жизни как изолированный акт, как акт, следующий за появлением необходимых для жизни условий, это скорее всего тот акт, тот момент, когда температура, воздух, вода, вообще земля приобрели соответствующие свойства, когда кислород, водород, углерод, азот вошли в такие соединения, которые вызвали существование органической жизни; этот момент вместе с тем был моментом, когда указанные вещества соединились для образования органических тел. Поэтому если в силу собственной природы с течением времени земля дошла до той ступени развития и культуры, когда она приняла, так сказать, человеческий вид, то есть вид, совместимый с существованием человека, соответствующий человеческому существу, то она оказалась в состоянии собственными силами вызвать появление человека.



\section{}

Мощь природы безусловно ограничена в отличие от божественного всемогущества, то есть силы человеческого воображения; она не в состоянии достигнуть всего в любое время и при любых обстоятельствах; ее достижения, ее действия связаны с известными условиями. Поэтому если в настоящее время природа не может порождать или не порождает организмов при помощи самопроизвольного зарождения, то из этого не следует, что и раньше она не была на это способна. В настоящее время состояние земли устойчиво; время революций прошло; они улеглись. Только вулканы представляют собой отдельные беспокойные головы, но они не имеют влияния на массы и поэтому не нарушают наладившегося порядка. Даже грандиознейшее вулканическое событие на памяти человечества, извержение Хорульо в Мексике, было только местным возмущением. Но ведь и человек проявляет необычные силы только в необычные времена, только в периоды высшего напряжения и движения он в состоянии сделать то, что вне данных условий ему прямо-таки не под силу; подобным образом растение только в известную пору, в период появления ростка, цветения и оплодотворения, образует тепло, сжигает углерод и водород, следовательно, проявляет животную функцию (превращает себя в животное, по словам Дюма), прямо противоположную обычным растительным отправлениям; так же точно земля обнаружила свою зоологическую, продуктивную силу только в эпоху своих геологических революций, в эпоху, когда все ее силы и все ее вещество были охвачены величайшим брожением, волнением и напряжением. Мы знаем природу лишь в ее теперешнем состоянии; следовательно, какое право мы имеем заключить, что несвойственное природе теперь вообще никогда не происходило, даже в совершенно другие времена при совершенно других условиях и обстоятельствах. Само собой разумеется, я не думаю, чтобы этими немногими словами можно было счесть разрешенной великую проблему происхождения органической жизни, но вышеизложенное достаточно для моей задачи; в самом деле, я даю лишь косвенное доказательство тому, что жизнь не может иметь иного источника, кроме природы. Что касается прямых, естественнонаучных доказательств, то мы, правда, еще очень далеки от цели, но достаточно продвинулись по сравнению с прежними временами, именно благодаря доказанному в новейшее время тождеству неорганических и органических явлений; во всяком случае продвинулись настолько, что можем счесть себя убежденными в естественном происхождении жизни, хотя способ этого происхождения нам неизвестен, а быть может, и навсегда останется неизвестным.



\section{}

Христиане не могли надивиться тому, что язычники почитали естественно возникшие существа за божественные; между тем их скорее следовало восхвалять за это, поскольку в основе этого почитания лежал совершенно правильный взгляд на природу. Возникать --- значит проявлять свою индивидуальность; индивидуальные существа возникли; между тем общие, лишенные индивидуальности, основные вещества, основные элементы природы никогда не возникали; никогда не возникала и материя. Но индивидуальное существо по качеству есть более высокое, более священное существо по сравнению с тем, что лишено индивидуальности. Конечно, рождение есть нечто постыдное, смерть --- мучительна; но кто не хочет иметь начала и конца, должен отказаться от звания живого существа.

Вечность ведь исключает жизненность, жизненность исключает вечность. Правда,
индивидуум предполагает  другое  порождающее существо, но  вследствие  этого
порождающее  существо стоит  не над, а под  существом  порожденным. Конечно,
порождающее существо есть причина бытия, и постольку оно --- первосущество, но
вместе с тем его можно также рассматривать как простое  средство и вещество,
как основу бытия другого существа и  постольку существо подчиненное. Ребенок
питается своей матерью, обращает на благо  себе ее силы и соки, румянит свои
щеки  ее  кровью.  И  ребенок  составляет гордость своей  матери, она ставит
ребенка   выше  себя,   подчиняет   существованию   и  благу   ребенка  свое
существование, свое  благо: даже самка животного жертвует собственной жизнью
для жизни  своих  детенышей. Величайшее  унижение для всякого существа --- его
смерть, но источник  смерти  коренится  в акте  рождения. Рождать  ---  значит
унижать себя,  отдаваться будням,  растворяться  в  массе, жертвовать  своей
индивидуальностью,  своей исключительностью  другим  существам.  Нет  ничего
более противоречивого,  извращенного и бессмысленного,  чем мысль о том, что
естественные   существа   рождены   высочайшим,   совершеннейшим,   духовным
существом. Так же точно  в соответствии с этим процессом, поскольку творение
есть образ творца, дети рода человеческого должны были бы  появляться не  из
матки  --- органа, находящегося  в нижней  части тела  матери,  но  из высшего
органа тела --- из головы.
\section{}

Древние греки сводили все источники, колодцы, потоки, озера и моря к океану, к мировому потоку или мировому морю, а древние персы считали, что все земные горы произошли от горы Альборди. Не то же ли самое выведение всех существ из одного совершенного существа? Такое выведение определяется тем же ходом мысли. Альборди --- такая же гора, как и все возникшие из нее горы, так же точно и божественное существо как первоисточник существ производных --- такое же существо, как и последние, не отличающееся от них по роду; гора Альборди выделяется среди других гор тем, что она обладает свойствами последних в высочайшей степени, иными словами, в степени, доведенной фантазией до высшей точки, до неба, выше солнца, луны и звезд; подобным же образом и божественное первосущество отличается от всех других существ тем, что оно обладает всеми их свойствами в наивысшей, безграничной, бесконечной степени. Но ведь изначальный поток воды не есть источник многообразных вод, первобытная гора не определяет собой других различных гор, так же точно первосущество не есть источник многих различных существ. Единство неплодотворно, плодотворен только дуализм, противоположность, различие. То, что созидает горы, не только отлично от гор, но и само по себе есть нечто в высшей степени разнородное; подобным же образом то, что образует воду, есть совокупность веществ, отличных не только от воды, но и различающихся между собой и даже противоположных друг другу. Подобно тому как остроумие, шутка, острота, суждение образуются и осуществляются лишь с помощью противоположностей, лишь в результате конфликта, так и жизнь возникает лишь благодаря конфликту разнообразных, даже противоположных веществ, сил и сущностей.



\section{}

<<Разве тот, кто создал ухо, не слышит? Разве не обладает зрением создавший глаз?>>. Это библейское или теистическое выведение наделенного слухом и зрением существа из существа, видящего и слышащего, или, на нашем современном философском языке, выведение духовного, субъективного существа из подобного ему духовного, субъективного существа покоится на тех же основах, равносильно библейскому объяснению дождя из небесных скопленных поверх всяких облаков водяных масс, равносильно взгляду персов на первую гору Альборди, породившую все другие горы, равносильно объяснению греков, что все источники и реки вытекают из одного океана. Вода --- из воды, но воды бесконечно обильной, всеохватывающей, гора --- от горы, но горы безмерной, всеохватывающей. Так же точно: дух --- от духа, жизнь --- от жизни, глаз --- от глаза, но от глаза, жизни и духа бесконечных, всеохватывающих.

\section{}

На вопрос, откуда родятся дети, нашим детям дают <<объяснение>>, будто их кормилица достает из колодца, в котором дети плавают наподобие рыб. Таково же теологическое <<объяснение>> возникновения органических, вообще естественных существ. Бог есть глубокий или прекрасный колодец фантазии, в котором заключены все реальности, все совершенства, все силы, где, следовательно, все вещи плавают в готовом виде, подобно рыбам; теология есть кормилица, извлекающая вещи из этого колодца, но главное лицо --- природа, мать, в муках рождающая ребенка, которого она носит под своим сердцем в продолжение девяти месяцев, --- совсем не принимается во внимание при этом объяснении, бывшем некогда младенческим, ныне ставшим ребячливым. Во всяком случае это объяснение красивее, уютнее, легче, удобопонятнее и убедительнее для божьих детей, нежели объяснение естественное, которое только постепенно, через бесчисленное количество препятствий пробивается из мрака на свет. Но ведь и тот способ, каким наши благочестивые отцы объясняли град, падеж скота, засуху и грозу действиями магов, волшебников и ведьм, гораздо <<поэтичнее>>, легче и убедительнее еще и теперь для необразованных людей, чем объяснение этих явлений из естественных причин.

\section{}

<<Возникновение жизни необъяснимо и непонятно>>; пусть будет так; но эта непонятность не дает тебе права для тех суеверных выводов, которые теология извлекает из пробелов человеческого знания; она не оправдывает твоих попыток выйти за пределы естественных причин, ибо ты можешь только сказать: я не могу объяснить жизнь из этих мне известных естественных явлений и причин или из них, каковыми я их знал доныне; но ты не имеешь права сказать: жизнь принципиально вообще необъяснима из природы --- ведь ты не имеешь основания считать, что ты исчерпал океан природы до последней капли; ты не имеешь права допущением воображаемых существ объяснять необъяснимое; ты не имеешь права объяснением, не дающим никакого объяснения, обманывать и вводить в заблуждение себя и других; ты не имеешь права свое незнание естественных материальных причин превращать в небытие таких причин; ты не имеешь права обожествлять, персонифицировать, объективировать свое невежество в такое существо, которое должно преодолеть это невежество, но которое на самом деле только выражает сущность твоего невежества, отсутствие положительных, материальных основ для объяснения. Ведь это нематериальное, нетелесное или бестелесное, внеприродное, внемировое существо, посредством которого ты объясняешь себе жизнь, не есть ли точное выражение отсутствия в твоем уме материальных, телесных, естественных, космических причин? Но вместо того, чтобы быть честным и скромным и сказать прямо: я не знаю причины, я не могу объяснить, у меня нет данных, нет материалов, --- ты этот недостаток, это отрицание, эту пустоту собственной головы превращаешь с помощью фантазии в положительные существа, в существа, которые представляют собой имматериальные, то есть не материальные, не естественные, существа, ибо ты не знаешь никаких материальных, никаких естественных причин. Впрочем, невежество удовлетворяется имматериальными, бестелесными, не природными существами, но неизменная спутница невежества, пышная фантазия, всегда занятая высшими, высочайшими и сверхвысочайшими существами, тотчас возводит эти несчастные создания невежества в ряд сверхматериальных, сверхъестественных существ.

\section{}

Взгляд, будто сама природа, мир вообще, вселенная имеет действительное начало, что, следовательно, некогда не было ни природы, ни мира, ни вселенной, есть убогий взгляд, который только тогда убеждает человека, когда его представление мира убого, ограниченно; это представление есть фантазия, бессмысленная и беспочвенная фантазия, будто некогда не было ничего действительного, ибо совокупность всей реальности, действительности и есть мир или природа. Все свойства или определения бога, превращающие его в предметное, действительное существо, представляют собой лишь отвлеченные от природы, природу предполагающие, природу выражающие свойства --- такие свойства, которые исчезают, как только кончается природа. Правда, у тебя остается сущность, совокупность таких свойств, как бесконечность, сила, единство, необходимость, вечность, даже тогда, когда ты отвлекаешься от природы, когда ты отвергаешь ее существование в мыслях или воображении, то есть когда ты закрываешь свои глаза, изгоняешь из себя все определенные чувственные образы естественных предметов, следовательно, представляешь себе природу не чувственной (не конкретной, по выражению философов). Но эта сущность, остающаяся за вычетом всех чувственных свойств и явлений, есть не что иное, как отвлеченная сущность природы, или природа в абстракции, природа в мыслях. И в этом отношении твое выведение природы или мира из бога --- не что иное, как выведение чувственной реальной сущности природы из ее абстрактной, мыслимой сущности, существующей только в представлении, только в мыслях; это выведение кажется тебе разумным потому, что ты всегда предпосылаешь абстрактное, всеобщее как ближайшее для мышления, следовательно, более для мысли высокое и раннее единичному, реальному, конкретному; между тем в действительности наоборот: природа предшествует богу, другими словами, конкретное предшествует абстрактному, чувственное --- мыслимому. В действительности, где все течет только естественным порядком, копия следует за оригиналом, образ --- за вещью, мысль --- за предметом, но в сверхъестественной, причудливой сфере теологии оригинал следует за копией, вещь следует за образом. Блаженный Августин говорит: <<Это удивительно, но это верно, что мы не могли бы знать этот мир, если бы он не существовал, но он не мог бы существовать, если бы бог его не знал>>. Это как раз значит: сначала мы познаем, мыслим мир, а потом он начинает существовать реально; да он существует лишь потому, что его помыслили, бытие есть следствие знания или мышления, оригинал есть следствие копии, сущность есть следствие образа.

\section{}

Если свести вселенную, или мир, к абстрактным определениям, если превратить мир в метафизическую вещь, следовательно в простой предмет мысли, и принять этот абстрактный мир за действительный, то логически неизбежно мыслить его конечным. Мир нам дан не мыслью, во всяком случае не метафизической и сверхприродной мыслью, абстрагирующей от реального мира и полагающей в этой абстракции свою подлинную высочайшую сущность; мир нам дан жизнью, созерцанием, чувствами. Для абстрактной, только мыслящей сущности нет света, ибо у нее нет глаз, нет теплоты, ибо у нее нет чувств, у нее вообще не существует никакого мира, ибо у нее нет органов для его восприятия, вообще для нее ничего не существует. Итак, мир нам дан только благодаря тому, что мы --- не логические или метафизические сущности, что мы --- другие существа, что мы больше, чем простые логики и метафизики. Но как раз этот плюс представляется метафизику минусом, это отрицание мышления представляется абсолютным отрицанием. Для метафизики природа есть только нечто противоположное духу --- <<его другое>>. Это исключительно отрицательное и абстрактное определение он превращает в положительную сторону природы, в ее сущность. Поэтому ему претит мыслить в качестве положительной сущности такой предмет или, скорее, небытие, которое сводится к простому отрицанию мышления, которое есть нечто мыслимое, но по природе своей чувственное, противоречащее мышлению, духу. Для мыслителя истинное существо есть мыслящая сущность; само собой понятно, что существо, которое не является мыслящей сущностью, не есть истинное, вечное, первоначальное существо. Духу претит помыслить нечто чуждое самому себе; он в согласии с самим собой, он в пределах своего бытия, когда он мыслит лишь самого себя (спекулятивная точка зрения) или во всяком случае (теистическая точка зрения) мыслит сущность, выражающую лишь сущность мышления. Такая сущность дана лишь через мышление и, стало быть, сама по себе есть только мыслимая, во всяком случае пассивная сущность. Таким образом, природа превращается в ничто. Тем не менее, она как-то существует, хотя она не может существовать и не должна существовать. Итак, как же метафизик объясняет ее наличность? Только мнимо добровольным, в действительности же противоречащим его глубочайшей сути, лишь принудительным самоотчуждением, самоотрицанием, самоотказом духа. Но если с точки зрения абстрактного мышления в ничто превращается природа, то, наоборот, с точки зрения реального миросозерцания исчезает этот создающий вселенную дух. При таком взгляде все дедукции --- мира из бога, природы из духа, физики из метафизики, действительности из абстракции --- оказываются логической игрой.

\section{}

Природа есть изначальный и основной объект религии, но даже там, где она оказывается непосредственным объектом религиозного почитания, как в естественных религиях, она не является объектом в качестве природы --- другими словами, в таком виде, в таком смысле, в каком мы ее рассматриваем с точки зрения теизма или философии и естествознания. Скорее природа первоначально представляется человеку объектом, как то, чем он сам является, как личное, живое, ощущающее существо; таков взгляд на природу, когда она созерцается глазами религии. Человек первоначально не отличает себя от природы, следовательно, не отличает и природы от себя; поэтому ощущения, которые в нем возбуждает объект природы, он непосредственно превращает в свойства самого объекта. Благоприятные, положительные ощущения и аффекты вызываются благим, благодетельным существом природы; отрицательные, вызывающие страдания ощущения --- жар, холод, голод, боль, болезнь --- причиняются злым существом или, во всяком случае, природой в недобром состоянии, в состоянии зложелательства, гнева. Таким образом, человек непроизвольно и бессознательно превращает природное существо в существо душевное, субъективное, то есть человеческое. Превращение это происходит необходимо, хотя эта необходимость только относительная, только историческая. Нет ничего удивительного, что человек затем уже вполне определенно, сознательно и намеренно превращает природу в религиозный объект, в объект молитвы, другими словами, в объект, который определяется человеческим чувством, его просьбами и служением. Человек уже тем сделал природу податливой, себе подчиненной, что он ее ассимилировал своим настроениям, что он ее подчинил своим страстям. Впрочем, необразованный, первобытный человек не только приписывает природе человеческие мотивы, влечения, страсти, он в естественных телах усматривает настоящих людей. Так, индейцы Ориноко принимают солнце, луну и звезды за людей, они говорят: <<Те, наверху находящиеся, --- это люди, как мы>>; патагонцы считают звезды за <<некогда существовавших индейцев>>; гренландцы видят в луне и звездах <<своих предков, которые при особых обстоятельствах были взяты на небо>>. Таковы же были мнения прежних мексиканцев, что солнце и луна, почитаемые в качестве богов, некогда были людьми. Обратите внимание! Так подтверждается высказанное в <<Сущности христианства>> положение, что человек в религии обращается лишь к самому себе, что его бог есть только его собственная сущность, подтверждается даже самыми грубыми, низшими видами религии, в которых человек почитает наиболее отдаленные, не схожие с ним предметы --- звезды, камни, деревья, даже клешни раков и раковины улиток, --- ведь он почитает их только потому, что он переносит в них самого себя, мыслит их в виде таких существ, каков он сам, или же считает, что они наполнены подобными существами. В связи с этим религия обнаруживает удивительное, но весьма понятное и даже неизбежное противоречие: в то время как с теистической или антропологической точки зрения она потому человеческое существо почитает за божественное, что оно ей представляется существом, отличным от человека, существом нечеловеческим, наоборот, с натуралистической точки зрения она нечеловеческое существо потому почитает за божественное, что оно ей представляется существом человеческим.


\section{}

Изменчивость всей природы, именно в явлениях, в наибольшей степени заставляющих человека чувствовать свою зависимость от нее, есть главное основание, почему природа представляется человеку в виде человеческого, наделенного волей существа и составляет для него предмет религиозного почитания. Если бы солнце непрерывно стояло на небе, оно никогда бы не зажигало в человеке огня религиозного аффекта. Человек только тогда преклонил свои колени перед ним, охваченный радостью при неожиданном возвращении солнца, когда оно исчезло из его глаз, обрекши человека на ночные страхи, а затем вновь появилось на небе. Так, древние апалачи во Флориде приветствовали хвалебными гимнами солнце при его восходе и закате и вместе с тем умоляли его вновь вернуться в надлежащее время и обрадовать их своим светом. Если бы земля неизменно приносила плоды, отпало бы основание для празднеств, связанных с посевом и жатвой. Только благодаря тому, что природа то открывает свое лоно, то скрывает его, плоды ее представляются добровольными, обязывающими к благодарности дарами. Только непостоянство природы делает человека мнительным, смиренным, религиозным. Неизвестно, будет ли завтра погода благоприятствовать моему предприятию, неизвестно, пожну ли я то, что посеял; итак, я не могу рассчитывать и надеяться на дары природы, как на уплату дани или неизбежное следствие. Но там, где кончается математическая достоверность, там начинается теология --- так это происходит даже в наши дни со слабыми головами. Религия есть созерцание необходимого, как чего-то произвольного, добровольного, в отдельных случайных явлениях. Противоположное настроение, настроение иррелигиозности и безбожия, обнаруживает циклон у Эврипида, говоря: <<Земля обязана, хочет она того или нет, выращивать траву для пропитания моего стада>>.


\section{}

Жертвоприношение, главный акт естественной религии, коренится в чувстве зависимости от природы, в соединении с представлением природы как сознательно действующего, индивидуального существа. Зависимость от природы особенно остро воспринимается, когда мы в ней нуждаемся. Нужда есть чувство или выражение коего ничтожества без помощи природы; но с нуждой неразрывно связано наслаждение --- противоположное чувство, чувство моей самости, моей самостоятельности в отличие от природы. Поэтому нужда богобоязненна, покорна, религиозна, наслаждение же высокомерно, не знает бога, не отличается почтительностью, легкомысленно. И это легкомыслие или, во всяком случае, бесцеремонность наслаждения практически необходима для человека, этой необходимостью определяется его жизнь, но она находится в прямом противоречии с теоретическим почтением человека перед природой как существом живым, эгоистическим, восприимчивым в человеческом смысле этого слова; существо это, подобно человеку, ревниво оберегает свои права. Этот захват или использование природы представляется поэтому человеку как бы нарушением чужого права, как бы присвоением чужого имущества, преступным деянием. И вот, чтобы успокоить свою совесть и ублаготворить объект, оскорбленный с человеческой точки зрения, чтобы показать, что он обворовал его по нужде, а не из заносчивости, человек урезывает свое наслаждение, возвращает объекту кое-что из его отнятой собственности. Так, греки воображали, что душа срубленного дерева --- дриада приносит жалобу и взывает к судьбе, чтобы она отомстила злодею. Так, ни один римлянин не решился бы на своем участке вырубить рощицу без того, чтобы не принести в жертву молодую свинью для умилостивления бога или богини этой рощи. Так, остяки, убив медведя, вешают его шкуру на дерево, оказывают ей всяческие почести и не за страх, а за совесть просят прощения у медведя за то, что они его убили. <<Они думают, что этим они вежливо отклоняют тот вред, который им мог бы причинить дух этого животного>>. Так, североамериканские племена подобными обрядами стремятся снискать милость у теней убитых ими зверей. Так, для наших предков бузина была священным деревом; когда же им приходилось срубать бузину, то предварительно они обычно произносили молитву: <<Госпожа бузина, удели нам от твоего дерева, тогда я дам тебе и от своего, когда оно вырастет в лесу>>. Так, филиппинцы просили разрешения у равнин и гор, когда им предстояло по ним путешествовать, и считали преступлением срубить какое-нибудь старое дерево. Так же брамин едва решается пить воду и ступать ногами по земле, потому что каждый шаг, каждый глоток воды может вызвать страдания и смерть чувствующих существ, растений и животных; поэтому он должен каяться, <<чтобы искупить смерть тварей, которых он и днем и ночью может не намеренно уничтожить>>. Сюда также относятся многие правила поведения, которые в древних религиях человек должен был соблюдать в отношении природы, чтобы ее не осквернить и не оскорбить. Так, например, ни один служитель Ормузда не смел касаться земли босыми ногами, ибо земля священна; ни один грек не осмеливался переходить через реку с невымытыми руками.


\section{}

В жертвоприношении становится наглядной и сосредоточивается вся сущность религии. Жертвоприношение коренится в чувстве зависимости --- в страхе, сомнении, неуверенности в успехе, в будущем, в угрызениях совести из-за совершенного греха; а результат, цель жертвоприношения лежит в самочувствии --- в храбрости, наслаждении, уверенности в успехе, свободе и блаженстве. Рабом природы приступаю я к жертвоприношению, а после жертвоприношения я чувствую себя хозяином природы. Поэтому корень религии --- в чувстве зависимости от природы; упразднение этой зависимости, освобождение от природы составляет цель религии. Другими словами, божественность природы в самом деле есть источник, основа религии, а именно всяческой религии, также и христианской; конечной же целью религии является обожествление человека.



\section{}

Предпосылка религии заключается в противоположности или противоречии между хотением и умением, между желанием и исполнением, между намерением а осуществлением, между представлением и действительностью, между мышлением и бытием. В хотении, желании, представлении человек не ограничен, свободен, всесилен, он --- бог, а в умении, осуществлении, в действительности он обусловлен, зависим, ограничен, он --- человек, человек в смысле конечного, противоположного богу существа. <<Человек предполагает, а бог располагает>>, <<У человека свои планы, Зевс же делает по-своему>>. Мышление, хотение --- в моей власти, но то, чего я домогаюсь и что я мыслю, принадлежит не мне; это --- вне меня, это зависит не от меня. Религия стремится и ставит себе целью устранить это противоречие или эту противоположность; божественное же существо и есть то существо, где это противоречие разрешается, где оказывается возможным или, вернее, действительным все то, что возможно согласно моим желаниям и представлениям, но невозможно для меня по моим силам.



\section{}

Исконный, подлинный, коренной элемент религии заключается в том, что не зависит от человеческой воли и знаний; это дело бога. Апостол Павел говорит: <<Я насадил, Аполлон поливал; но возрастил бог. Посему и насаждающий и поливающий есть ничто, а все --- бог, взращивающий>>. Также Лютер: <<Мы должны\dots восхвалять бога и благодарить его, взращивающего нам хлеб, мы должны признать, что не нашей работой, но его благословением и его малостью произрастают зерно, вино и всевозможные плоды, доставляющие нам еду, питье и удовлетворение всяческих потребностей>>. Гезиод говорит, что усердный земледелец соберет богатую жатву, если Зевс обеспечит удачное завершение. Итак, вспахать поле, посеять, полить всходы зависит от меня; жатва же не в моей власти. Жатва --- в руке божией; отсюда поговорка: <<Без бога ни до порога>>. Но что такое бог? В первоначальном смысле не что другое, как природа или сущность природы, но как объект молитвы, как существо, которое мы просим, следовательно, как существо, проявляющее волю. Зевс --- причина или сущность метеорологических явлений природы; но это еще не составляет его божественный, его религиозный элемент, и для нерелигиозного человека дождь, гром, снег имеют причину. Он бог только потому и только в силу того он бог, что он --- владыка метеорологических явлений природы, что эти явления природы зависят от его усмотрения, составляют акты его воли. Итак, независимое от человеческой воли ставится религией в зависимое положение от воли божией, со стороны предмета (объективно); со стороны же человека (субъективно) оно делается зависимым от молитвы, ибо то, что зависит от воли, составляет предмет молитвы, есть нечто изменчивое, есть то, о чем можно просить. <<Сами боги могут оказаться послушными. Ладаном и смиренными обетами, возлиянием вина и благовониями смертный может их направить в другую сторону>>.

\section{}

Где человек возвысился над безграничной свободой выбора, над беспомощностью и случайностью подлинного \underline{фетишизма}, там во всяком случае предметом религии, исключительно или главным образом оказывается то, что составляет предмет целеустремленной деятельности и потребностей человека. Именно поэтому безусловно всеобщим и преимущественным религиозным почитанием пользовались те естественные существа, которые были наиболее необходимы человеку. А то, что составляет предмет человеческих потребностей и целеустремленной деятельности, и есть как раз объект человеческих желаний. Мне нужен дождь и солнечный свет, чтобы мой посев взошел. Поэтому-то при продолжительной засухе я желаю дождя, при продолжительном дожде --- солнца. Желание есть стремление, исполнение которого не в моей власти, оно есть воля, бессильная добиться желаемого; если я и добиваюсь желаемого и, как таковое, оно мне доступно, то оно может быть не в моей власти в данный момент, при данных обстоятельствах и условиях; если принципиально мое желание и осуществлено, то не так, как того хочет человек с религиозной точки зрения. Но то, что недоступно моему телу, вообще моим силам, то --- во власти самого моего желания. К чему я стремлюсь, чего я желаю, то я зачаровываю, одухотворяю своими желаниями. На старонемецком языке wiinschen (желать) --- значит зачаровывать. В аффекте человек полагает свою сущность за пределы самого себя, и только в аффекте, ибо в чувстве коренится религия; в аффекте он принимает безжизненное за жизненное, непроизвольное за произвольное, одухотворяет предмет своими вздохами, ибо, находясь в аффекте, он не может обращаться к бесчувственному существу. Чувство переходит за пределы, предписанные рассудком, оно льется через края человеческой природы, чувству слишком тесно в груди, оно должно перейти во внешний мир, превратив бесчувственное существо природы в существо сочувствующее. Природа, зачарованная человеческим чувством, ему соответствующая и с ним ассимилировавшаяся, следовательно, сама преисполненная всяческого чувства, и есть та природа, которая составляет предмет религии, божественное существо. Желание есть источник, есть самая суть религии, сущность богов есть не что иное, как сущность желания. Боги --- сверхчеловеческие и сверхприродные существа. Боги --- это те существа, от которых исходит благословение. Благословение есть результат, есть плод, цель действия, от меня независимая, но мною желаемая. Лютер говорит: <<Благословлять --- собственно, значит желать чего-то хорошего>>. <<Когда мы благословляем, мы только желаем добра; однако не можем дать того, чего мы хотим; но божие благословение приумножает и составляет силу>>. Это значит: люди --- существа, которые желают; боги --- такие существа, которые исполняют желания. Так, столь часто употребляемое в обычной жизни слово <<бог>> есть выражение желания. <<Дай тебе бог детей>> означает: я желаю тебе детей. Только в последних словах это желание выражено субъективно, не религиозно, по-пелагиански, а в первом случае --- объективно, следовательно, религиозно, по-августиновски. Но разве желания не сверхчеловеческие и не сверхприродные существа? Разве я, например, остаюсь еще человеком в своем желании, в своей фантазии, если я мечтаю стать бессмертным существом, избавившимся от оков земного тела? Нет! У кого нет желаний, у того нет и богов. Почему греки так подчеркивали бессмертие и блаженство богов? Потому что они сами не хотели быть смертными и лишенными блаженства. Где ты не слышишь жалобных песен о смертности и злоключениях человека, там ты не услышишь и хвалебных гимнов бессмертным и блаженным богам. Слезы сердца испаряются в небо фантазии, в туманный призрак божественного существа. Гомер выводит божества из мирового потока океана, но этот божественный поток в действительности есть лишь излияние человеческих чувств.


\section{}

Антирелигиозные явления в области веры всего нагляднее раскрывают происхождение и сущность религии. Таким антирелигиозным фактом, с горьким упреком подмеченным еще благочестивыми язычниками, является, например, то обстоятельство, что вообще люди только в несчастии прибегают к религии, только в несчастии обращаются к богу и думают о нем, но как раз этот факт приводит нас к источнику самой религии. Человек приходит к мучительному выводу, что он не может того, чего хочет, что у него связаны руки, когда он находится в несчастии, в нужде, своей ли собственной или чужой. Но расслабление двигательных нервов не есть одновременно расслабление чувствующих нервов; оковы моих телесных сил не оказываются оковами моей воли, моего сердца. Наоборот, чем больше у меня связаны руки, тем свободнее мои желания, тем сильнее моя жажда избавления, тем энергичнее мое стремление к свободе, тем сильнее мое желание не быть ограниченным. Чрезмерно возбужденная, сверхчеловеческая сила сердца и воли, взвинченная у людей до последней степени властью нужды, и есть божественная сила, не знающая ни нужды, ни границ. Боги в состоянии сделать то, чего хотят люди, другими словами, они реализуют заколы человеческого сердца. Чем люди являются по своей душе, таковы боги телесно; то, что людям доступно только в хотении, в воображении, в сердце, то есть только духовно (например, разом перенестись в отдаленное место), то находится в физической власти богов. Боги --- воплощенные, овеществленные, осуществленные желания человека, упраздненные естественные пределы человеческого сердца и воли, существа с неограниченной волей, существа, чьи телесные силы равны силе их хотений. Антирелигиозное проявление этой сверхъестественной религиозной силы есть волшебство у некультурных народов: здесь простая воля человека совершенно явно оказывается богом, который властвует над природой. Если израильский бог по требованию Иисуса Навина приказывает солнцу остановиться, если он посылает дождь по просьбе Илии, если христианский бог для доказательств своей божественной природы, то есть своей силы, выполняет все пожелания людей, одним своим словом успокаивает разбушевавшееся море, исцеляет больных, воскрешает мертвых, то здесь так же, как при волшебстве, простая воля, простое желание, простое слово объявляются силой, господствующей над природой. Разница лишь в том, что волшебник реализует цель религии антирелигиозно, а иудей, христианин --- религиозно, причем первый усматривает в себе то, что вторые переносят в бога; первый превращает в объект настойчивого желания, приказа то, в чем последние усматривают объект тихого, покорного волеизъявления, скромного желания, словом, первый действует через себя и для себя, а вторые --- через бога и с богом. Но известная поговорка: guod quis per alium fecit, ipse fecisse putatur, иначе говоря, то, что один делает при помощи другого, то ему приписывается в качестве собственного деяния --- находит здесь свое применение: то, что человек делает через бога, то в действительности делает он сам.


\section{}

Единственной задачей и целью религии (во всяком случае в первую голову и по отношению к природе) является превращение неизвестного и жуткого существа природы в знакомое, близкое существо, размягчение в пламени сердца неумолимой самой по себе, твердой, как железо, природы на пользу человека --- словом, религия ставит себе такую же цель, как просвещение или культура; их стремление сводится к тому, чтобы сделать природу в теоретическом отношении понятной, в практическом отношении --- податливой, соответствующей человеческим потребностям, однако с тем различием, что культура достигает своей цели с помощью средств, а именно с помощью заимствованных из самой природы средств, религия же --- без средств, иначе говоря, с помощью сверхъестественных средств --- молитвы, веры, таинств, волшебства. Поэтому все, что при развитии культуры человеческого рода стало делом образования, самодеятельности, антропологии, то было первоначально делом религии или теологии; таковы, например, право (ордалии, испытания кровью, судебные оракулы германцев), политика (оракулы греков), врачевание, которое еще в наши дни у некультурных народов является функцией религии. Поэтому религия является средством просвещения человечества в суровые времена и для диких народов, но в эпоху просвещения религия потакает дикости, архаичности; она --- враг просвещения. Правда, культура всегда отстает от пожеланий религии; она не может уничтожить ограниченность человека, коренящуюся в его существе. Так, например, культура дает нам указания на средства долгой жизни, но не доставляет нам бессмертия. Последнее остается безграничным, неосуществимым желанием религии.


\section{}

В естественной религии человек обращается к объекту, который прямо-таки противоречит подлинному устремлению и смыслу религии. Действительно, человек жертвует своими чувствами для сущности, самой по себе бесчувственной, отдает ей, чуждой разума, свой ум, он ставит над собою то, что он хотел бы иметь под собой, он находится в услужении у того, над чем он хочет властвовать, он с почтением относится к тому, что он по существу ненавидит, он взывает о помощи к тому, от кого он ищет защиты. Так греки приносили жертвы ветрам на Титане, чтобы умилостивить их ярость; так римляне посвятили храм лихорадке, чтобы ее обезвредить; так тунгусы во время эпидемий благоговейно и с торжественными поклонами умоляют болезнь, чтобы она миновала их юрты (Паллас); так жители Гвинеи приносят жертвы бурному морю, чтобы побудить его успокоиться и не мешать им ловить рыбу; так индейцы при приближении грозы или бури обращаются к манито (духу, божеству, существу) воздуха, а при путешествии по воде --- к манито вод, чтобы он отклонил от них все опасности; так вообще многие народы определенно почитают не доброе, а злое существо природы, во всяком случае оно им представляется злым. Сюда же относится почитание вредных животных. В естественных религиях человек обращает свои любовные речи к статуе, к трупу; поэтому нет ничего удивительного в том, что он прибегает к самым, отчаянным, безумным средствам, чтобы быть выслушанным, нет ничего удивительного, что он становится бесчеловечным, чтобы очеловечить природу, что он даже проливает человеческую кровь, чтобы внушить природе человеческие чувства. Так, северные германцы определенно верили, что <<кровавые жертвы могут наделить человеческой речью и восприятием деревянных божков, также заставить говорить и пророчествовать камни, почитаемые в зданиях, где приносятся кровавые жертвы>>. Но тщетны все эти попытки оживить мертвую природу: природа отвечает молчанием на жалобы и вопросы человека; она беспощадно отбрасывает его обратно к самому себе.

\section{}

Хотя человек чувствует и представляет себя с религиозной точки зрения ограниченным, эти границы на самом деле являются только границами представления и фантазии и в действительности вовсе не являются доказательством роковой ограниченности человека, потому что они необходимо определяются существом дела, коренятся в природе вещей; таково, например, ограничение, что человек не знает будущего, что он не вечен, что он не пользуется непрерывным и неомраченным счастьем, что его тело обременено тяжестью, что он не может летать, как боги, не может распространять гром, как Иегова, не может по собственному почину увеличивать свой облик или делать его невидимым, не может, подобно ангелу, жить без чувственных потребностей и влечений, --- словом, не в состоянии достигнуть всего того, чего он хочет или желает, --- так же точно бесконечное божественное существо, неограниченное этими пределами, есть лишь сущность нашего представления, нашей фантазии и нашего настроения или чувства, находящегося во власти фантазии. Поэтому, что бы ни было предметом религии, пусть это будут даже раковины улитки или кремень, они окажутся предметом религии лишь как сущность, определяемая чувством, представлением, воображением. Это лежит в основании утверждения, что люди почитают не камни, не зверей, не деревья, не реки, как таковые, но обитающие в них божества, манито, духи. Но эти духи природных существ не что иное, как представления, как их отображения, или: это --- представляемые существа, воображаемые существа в отличие от этих же существ, но рассматриваемых как нечто чувственное, доподлинное. Точно так же духи умерших не что иное, как представления и образы умерших, не исчезающие из памяти, --- они были когда-то действительными существами, а теперь только представляются таковыми; религиозный, то есть необразованный, человек не различает между предметом и представлением этого предмета, почему эти существа кажутся ему реальными и самостоятельными. Совершенно ясен и очевиден благочестивый, непроизвольный самообман верующего человека, если он обнаруживается в естественной религии, ибо человек сам здесь вкладывает в свой религиозный объект и глаза и уши, он знает, он видит, что это сделанные каменные или деревянные глаза и уши, и все же он верит, что это действительные глаза и уши. Итак, верующий человек обладает глазами только для того, чтобы не видеть, чтобы быть совсем слепым, он обладает разумом только для того, чтобы не мыслить, чтобы быть совсем глупым. Естественная религия есть наглядное противоречие между представлением и действительностью, между фантазией и истиной. Что в действительности --- мертвый камень или чурбан, то в понимании естественной религии --- живое существо, что по всей видимости не бог, а нечто совсем другое, то невидимо, согласно вере --- бог. Поэтому естественная религия находится в постоянной опасности горчайшего разочарования. В самом деле, достаточно удара топором для обнаружения, например, что из боготворимых этой религией деревьев не течет никакой крови, что, следовательно, в них не пребывает никакое живое божественное существо. Как же теперь религия ускользает от этих грубых противоречий и разочарований, которым она себя подвергает своим почитанием природы? Только тем, что она превращает свой предмет в нечто невидимое, вообще нечувственное, в существо, которое есть только предмет веры, представления, воображения, словом, духа, итак, само по себе это есть духовное существо.


\section{}

Из чисто физического существа человек становится существом политическим, вообще он становится чем-то отличным от природы, сосредоточенным на самом себе; так же точно его бог из чисто физического существа становится существом политическим, отличным от природы. Только своим объединением с другими людьми в общине человек приходит к разграничению своей сущности от природы и, следовательно, --- к богу, отличному от природы; в этой общине предметом его сознания и чувства зависимости является сила закона, общественного мнения, чести, добродетели, то есть отличные от естественных сил, данные лишь в мысли и представлении политические, моральные, абстрактные силы; в общине физическое бытие человека подчинено его человеческому, гражданскому или моральному существованию; в общине естественная сила, власть над смертью и жизнью низводится до атрибута и орудия политической и моральной власти. У Гезиода прямо говорится: и молва (аов, слух, общественное мнение) --- тоже божество. Зевс --- бог молнии и грома, но это грозное оружие он держит в своих руках лишь для того, чтобы сокрушить нарушителей его законов, клятвопреступников, насильников. Зевс --- отец царей, <<цари происходят от Зевса>>. Таким образом, молнией и громом Зевса поддерживается власть и достоинство царей. Впрочем, следует первоначальных царей отличать от законных царей. Если не учитывать исключительных случаев, то последние --- обычные, сами по себе незначительные лица, первыми же были необыкновенные, выдающиеся исторические личности. Поэтому обоготворение исключительных людей после их смерти есть явление нормальное --- переходная ступень от религий естественных в собственном смысле слова к религиям мифологическим и антропологическим, хотя это обоготворение может встречаться и наряду с почитанием природы. Почитание выдающихся людей как богов ни в какой мере не свойственно только баснословным временам; так, уже в век христианства шведы обожествляли своего короля Эриха и приносили ему жертвы после его смерти. В законах Ману читаем: <<Царь, подобно солнцу, опаляет глаза и сердца, поэтому ни одно человеческое земное существо не может даже взглянуть на него. Он --- огонь и воздух, он --- солнце и луна, он --- божественный судья. Огонь пожирает только отдельных людей, беззаботно к нему приближающихся, огонь же царя, если он гневен, сжигает целую семью, со всем ее скотом и имуществом\dots В его мужестве обитает победа, в его гневе --- смерть>>. Так же точно израильский бог громом и молнией повелевает своим избранным ходить всеми путями, которые он им заповедал, <<чтобы они могли жить, чтобы они могли благоденствовать и долгоденствовать в стране>>. Так власть природы как таковая и чувство зависимости от нее исчезают перед лицом власти политической или моральной. Раба природы ослепляет блеск солнца, так что он, как качинский татарин, ежедневно его молит: <<не убивай меня>>; между тем раб политический ослепляется блеском царского звания до такой степени, что он падает перед ним ниц, как перед божественной силой, от которой зависит жизнь и смерть. Даже среди христиан римские императоры титуловались: <<ваша божественность>>, <<ваша вечность>>. Даже в наши дни среди христиан <<святейшество>> и <<величество>> --- эти титулы и атрибуты божества оказываются титулами и атрибутами королей. Правда, христиане оправдывают это политическое идолопоклонство толкованием, будто король --- лишь заместитель бога на земле, бог есть царь царей, но это оправдание --- простои самообман. Уже не говоря о том, что власть короля есть власть в высшей степени ощутительная, непосредственная, чувственная, самодовлеющая, власть же царя царей только опосредствованная, только воображаемая, --- бог определяется и рассматривается в качестве правителя вселенной, в качестве царской или вообще политической власти лишь там, где личность короля до такой степени заполняет, определяет человека и овладевает им, что он ее считает за высшее существо. Ману говорит: <<В начале времен Брама образовал для собственного употребления гения наказания с телом из чистого света в виде собственного сына, даже как основоположника уголовного законодательства, как защитника всего сотворенного. Из страха перед наказанием эта вселенная может наслаждаться своим счастьем>>. Так человек даже наказания своего уголовного права превращает в божественные силы, владычествующие над миром, превращает суровый уголовный порядок в строй вселенной, уголовный кодекс --- в кодекс природы. Нет ничего удивительного, что он ближайшим образом приобщает природу к своим политическим злоключениям и страстям, даже строй природы делает зависимым от строя своего королевского трона или папского престола. Что важно для него, то, разумеется, важно и для всех других существ; что туманит его очи, то затуманивает и сияние солнца; что задевает его сердце, то приводит в движение также небо и землю; его сущность для него --- универсальная сущность, сущность вселенной, сущность сущностей.


\section{}

Где причина того, что у Востока нет такой живой развивающейся истории, как у Запада? Она в том, что на Востоке человек ради человека не забывает природы, ради блеска человеческих глаз не забывает блеска звезд и драгоценных камней, ради риторического <<грома и молнии>> не забывает молнии и грома метеорологических, ради течения повседневных событий не забывает хода солнца и звезд, ради смены мод не забывает смены времен года. Правда, восточный человек падает в прах перед блеском царской политической власти и сана, но этот блеск сам есть лишь отблеск солнца и луны; для него царь не есть земной, человеческий объект, но небесное, божественное существо. Перед лицом бога человек исчезает. Только там, где земля обезбоживается, боги поднимаются на небо, из действительных существ превращаются в существа лишь воображаемые; только там перед людьми открывается поприще для деятельности, только там они в качестве людей могут показать себя и играть известную роль. Восточный человек стоит в таком же отношении к западному, как сельский житель к горожанину. Первый зависит от природы, второй --- от человека; первый живет по барометру, второй руководствуется курсом ценных бумаг, первый ориентируется по неизменным знакам зодиака, второй --- по непрестанно меняющимся признакам чести, моды и общественного мнения. Поэтому только у горожан есть история. Только, так сказать, человеческое тщеславие составляет руководящую нить истории. К историческим деяниям способен лишь тот, кто силу природы приносит в жертву власти мнения, кто свою жизнь жертвует своему имени, телесное бытие --- будущей славе и мнению потомства.


\section{}

Греческий комик Анаксандрит, по Атенею, сказал египтянам следующее: <<Я не гожусь для вашего общества, наши нравы и законы не одинаковы: вы поклоняетесь быку, которого я приношу в жертву богам; для вас угорь --- великое божество, для меня это лакомое блюдо; вы избегаете свинины --- я ем ее с аппетитом; у вас в почете собака --- я бью ее, если она у меня стянет кусок; вы в ужасе, если чего недостает кошке --- я же с удовольствием снимаю с нее шкуру; вы придаете значение землеройке --- для меня она ничто>>. Эта речь прекрасно характеризует противоположность между несвободным и самостоятельным, то есть между религиозным и антирелигиозным, свободным человеческим взглядом на природу. Там природа --- предмет почитания, здесь --- наслаждения; там --- человек для природы, здесь --- природа для человека; там --- она цель, здесь --- средство; там она стоит над человеком, здесь она ему подчинена. В настоящих строках я отождествляю точку зрения греков и израильтян, тогда как в <<Сущности христианства>> я их противопоставляю. Какое противоречие! Ничуть; вещи различные, если их сравнивать между собой, в свою очередь совпадают, будучи противопоставлены чему-то третьему. Впрочем, к наслаждению природой прежде всего относится также эстетическое, теоретическое наслаждение. Поэтому там человек эксцентричен, он выходит за свои пределы, он вне своей определенной сферы, указывающей ему лишь на него самого; здесь он, наоборот, рассудителен, спокоен, у себя дома, в полном самосознании. Там вполне последовательно человек для доказательства своего естественно-религиозного смирения снижается до совокупления с животными (Геродот); здесь, наоборот, человек возвышается в упоении своей силы и достоинства до смешения с богами. Это должно удостоверить, что даже в небесных божествах течет только человеческая кровь, что особенная, эфирная кровь богов --- лишь поэтический образ, не выдерживающий критики в действительности, в практике.



\section{}

В каком виде природа, вселенная проявляется для человека, такова она, разумеется, для него по его представлению. Его чувства, его представления непосредственно и бессознательно служат ему мерилом истины и действительности, которая ему кажется в том виде, каков он сам. Когда человек приходит к сознанию, что для его жизни кроме солнца и луны, неба и земли, огня и воды, растений и животных необходимо приложение, и именно правильное приложение собственных сил; что <<смертные несправедливо во всем обвиняют богов, сами, наперекор судьбе, своим безумством вовлекая себя в беду>>; что порок и глупость имеют своим последствием болезнь, несчастье, смерть, а добродетель и мудрость --- здоровье, жизнь и счастье; что, следовательно, ум и воля являются силами, определяющими судьбу человека; когда, таким образом, человек становится мыслящим, разумным существом, не подчиняясь, как это делает дикий человек, случайным, мгновенным впечатлениям и аффектам, но руководствуясь принципами, правилами благоразумия и разумными законами, --- тогда и природа, вселенная является и становится для него определенным существом, зависящим от ума и воли.



\section{}

Если человек, наделенный волей и умом, возвышается над природой, становится супранатуралистом, то и бог становится сверхъестественным существом. Если человек становится властителем <<рыб в море, птиц в поднебесье, скота и всей земли и всех пресмыкающихся на земле>>, то для него власть над природой оказывается высшим представлением, человек становится высшим существом; объектом его почитания, следовательно его религией, будет человек как творец природы, ибо неизбежным следствием или, скорее, предпосылкой владычества является творение. Если владыка природы не есть вместе с тем ее зиждитель, то она оказывается по своему возникновению и бытию независимой от него, следовательно, его власть ограничена и недостаточна; в самом деле, ведь если бы он мог создать природу, почему же он ее не создал? Власть его над ней оказывается узурпированной, а не прирожденной, не правомерной. Только то, что я созидаю, что я делаю, находится вполне в моей власти; право собственности распространяется только на мною сделанное. Ребенок --- мой, потому что я его отец. Итак, только творением оправдывается, реализуется, исчерпывается владычество. Языческие боги уже были действительно владыками природы, но не творцами ее, поэтому они --- конституционные, ограниченные, введенные в известные рамки, не абсолютные монархи природы, иначе говоря, язычники еще не были абсолютными, безусловными, радикальными сторонниками всего сверхъестественного.


\section{}

Теисты объявили учение о единстве бога сверхчувственным, основанным на откровении учением по его происхождению, не учитывая, что человек является источником монотеизма, что единство бога коренится в единстве человеческого сознания и духа. Мир раскрывается перед моим взором в бесконечном многообразии и разнообразии, но вместе с тем мой дух, моя голова охватывает все эти бесконечные и разнообразные предметы: солнце, луну и звезды, небо и землю, близкое и отдаленное, наличное и отсутствующее. Монотеизм ставит во главу мира и провозглашает причиной его эту сущность человеческого духа или сознания, сущность, столь удивительную для религиозного, то есть некультурного, человека, сущность сверхъестественную, не связанную ни с какими временными или пространственными пределами, не ограниченную никаким определенным родом вещей, заключающую в себе все предметы, все существа без того, чтобы самой быть предметом или видимым существом. Бог называет, бог мыслит мир, и мир существует. Бог повелевает ему не быть, бог не мыслит и не желает его бытия, и его нет; другими словами: я в своем мышлении, силой своего представления и воображения, могу все вещи, а следовательно и весь мир, по произволу вызывать и уничтожать, создавать и обрекать на гибель. Бог, создавший мир из ничего и вновь по своей воле превращающий его в ничто, есть не что иное, как сущность человеческой способности абстракции и силы воображения; согласно ей я по своему желанию могу себе представить мир существующим или несуществующим, могу утвердить и могу уничтожить его бытие. Это субъективное небытие, это отсутствие мира в представлении превращается монотеизмом в его объективное, действительное небытие. Политеизм, вообще естественная религия, превращает действительные существа в представляемые, воображаемые существа; монотеизм же превращает представляемые существа, вымыслы, мысли в существа действительные или, вернее, сущность силы представления, мышления и воображения --- в реальную, абсолютную, высшую сущность. Один богослов говорит, что власть бога простирается настолько, насколько простирается человеческая способность представления. Но где предел представляющей способности? Разве есть что-нибудь невозможное для силы воображения? Я в силах помыслить, как нечто несуществующее, все реальное и как существующее --- все нереальное; так, я могу <<этот>> мир представить себе как несуществующий, а бесконечное число других миров --- как существующие. То, что представляется действительным, есть возможное. Бог же есть существо, для которого нет ничего невозможного; он по своей силе --- творец бесчисленных миров, средоточие всех возможностей, всего, что можно себе представить; другими словами, он не что иное, как сущность человеческой способности воображения, мышления и представления, сущность, ставшая действительной, предметной, служащая объектом мысли и представления в качестве действительного и даже наидействительнейшего, абсолютного существа.


\section{}

Подлинный теизм или монотеизм возникает только тогда, когда человек относит природу лишь к себе и превращает это отношение в ее сущность, следовательно, в себе усматривает конечную цель, делает себя центральной и объединяющей точкой природы; это происходит потому, что природа, как безвольный и бессознательный предмет, используется им не только для его необходимых, органических, жизненных отправлений, но также для его произвольных, сознательных целей, действии и наслаждений. Один отец церкви определенно называет человека связью всех вещей, поскольку бог в нем хотел сосредоточить вселенную в единство, поэтому все объединяется в человеке, как в цели, все стремится к его пользе. Во всяком случае, и человек, как самое индивидуальное существо природы, есть ее завершение, но не в противоестественном и супранатуралистическом смысле телеологии и теологии. Если природа полагает свою цель вне себя, то и ее основание и начало по необходимости тоже --- вне ее; если она существует только для другого существа, то она необходимо и происходит от другого существа, а именно от того существа, намерением или целью которого при создании был человек как существо, наслаждающееся природой и обращающее ее в свою пользу. Начало природы восходит к богу лишь в том случае, когда завершение ее оказывается в человеке; другими словами, учение: бог --- творец мира, имеет свое основание и свой смысл лишь в учении: человек есть цель творения. О! Если вы стыдитесь верить, что мир сотворен, сделан для человека, то стыдитесь также того вероучения, что вообще мир сотворен, сделан. Если написано: <<вначале бог сотворил небо и землю>>, то там же еще написано: <<бог создал два великих светила, а также звезды и поместил их на тверди небесной, чтобы они освещала землю и управляли днем и ночью>>. О! Если вы веру в человека как цель природы называете человеческим высокомерием, то назовите также человеческим высокомерием веру в творца природы. Только тот свет, который светит ради человека, есть свет теологии; только тот свет, который имеется лишь ради существа, наделенного зрением, предполагает в виде причины зрячее существо.


\section{}

<<Духовное существо>>, которое человек предпосылает природе как ее возглавляющее, обосновывающее и созидающее существо, в действительности есть духовное существо самого человека, но это существо представляется чем-то самостоятельным, отличным от человека и с ним несравнимым, потому что человек превращает его в причину природы, в причину действий, которые не могут быть вызваны человеческим духом, человеческой волей и умом, что, следовательно, человек с этим духовным, человеческим существом одновременно связывает отличную от человеческого существа сущность природы. Эта связь или смешение <<морального>> и <<физического>> существа, существа человеческого и нечеловеческого, порождает третье существо, которое не есть ни природа, ни человек, но, как амфибия, причастно и тому и другому; именно благодаря этой природе сфинкса оно является кумиром мистики и умозрения. Благодаря божественному духу растет трава, благодаря ему в материнском теле развивается ребенок, благодаря ему солнце не выходит из своей орбиты и неизменно движется, благодаря ему вздымаются горы, веют ветры, и море остается в своих пределах. Что такое человеческий дух в сравнении с этим духом! Как он мелок, как он ограничен, как он ничтожен! Поэтому, если рационалист оставляет мысль об очеловечении бога, о соединении божественной и человеческой природы, то это происходит главным образом потому, что ему позади его бога мерещится природа, а именно природа в том виде, какой она открылась человеческому глазу через астрономический телескоп. Он в негодовании восклицает: как могло это громадное, бесконечное, всеобъемлющее существо, которое находит достойное себе выражение и действие только в великой, бесконечной вселенной, --- как могло бы оно ради человека сойти на землю, которая перед лицом грандиозности и полноты мирового целого превращается в ничто! Что за недостойный, мелкий, чисто <<человеческий>> взгляд! Сосредоточивать бога на земле, снижать бога до человека равносильно желанию вместить в капле океан, в перстне --- кольцо Сатурна. Разумеется, это --- наивное представление, что сущность мира ограничивается землей или человеком, что природа существует лишь ради последнего, что солнце светит только ради человеческого глаза. Но ты, близорукий рационалист, не видишь, что то, что в тебе сопротивляется соединению бога с человеком, что заставляет тебя признать это соединение бессмысленным противоречием, не есть представление о боге, но представление о природе или мире; ты не видишь, что объединяющая точка, что третье ближайшее понятие между богом и человеком не есть существо, которому ты косвенно или непосредственно приписываешь силу и действия природы, но, скорее, то существо, которое обладает зрением и слухом, потому что ты видишь и слышишь, обладает сознанием, умом и волей, потому что ты обладаешь ими; итак, это есть то существо, которое ты отличаешь от природы, поскольку и как ты сам себя от нее отличаешь. Итак, что ты смог бы возразить, если бы это человеческое существо в конце концов предстало перед твоим взором в облике действительного человека? Как ты можешь отказаться от вывода, если ты признаешь основание этого вывода? Как ты будешь отрицать сына, если ты признаешь отца? Если для тебя богочеловек --- плод человеческой фантазии и самообожествления, то и в творце природы усмотри создание человеческой фантазии и желание человека возвыситься над природой. Если ты хочешь обладать существом, не наделенным никакими человекообразными признаками, без всяких человеческих привнесений, будут ли то привнесения ума, сердца или воображения, то будь настолько решителен и последователен, чтобы отказаться от бога вообще, чтобы апеллировать и опереться на чистую, незапятнанную, безбожную природу как последнюю основу своего бытия. Пока ты допускаешь отличие бога от природы, до тех пор ты оставляешь в силе человеческое отличие, до тех пор ты в этом первосуществе обожествляешь лишь собственную сущность; в самом деле: для отличия от человеческого существа у тебя нет и ты не знаешь другой сущности, кроме природы; точно так же и наоборот: у тебя нет и ты для отличия от природы не знаешь никакого другого существа, кроме существа человеческого.

\section{}

Взгляд на человеческое существо, как на существо, отличное от человека, как на существо предметное, другими словами: опредмечивание человеческого существа опирается на предпосылку очеловечения предметного существа, отличного от человека. Это есть взгляд на природу, как на человеческое существо. Поэтому с этой точки зрения творец природы есть не что иное, как сущность природы, помощью абстракции отличенная и отвлеченная от действительной природы, от природы как предмета чувств, --- сущность природы, помощью силы воображения превращенная в человеческое или человекоподобное существо, популяризированная, антропоморфизированная или персонифицированная. Поэтому воля и ум представляются человеку основными силами или причинами природы лишь потому, что непроизвольные действия природы в свете его рассудка кажутся ему преднамеренными, целенаправленными, а следовательно, природа --- разумным существом или во всяком случае --- чистым предметом разума. Как все видимо для солнца --- бог солнца <<Гелиос все видит и слышит>>, --- потому что человек все воспринимает в свете солнца, --- так же точно само по себе все есть нечто мыслимое, потому что оно мыслится человеком, есть создание разума, потому что оно является для него объектом разума. Поскольку человек измеряет звезды и расстояния их друг от друга, постольку они сами по себе измерены; раз для познания природы человек применяет математику, то, значит, она была применена и при создании природы; раз человек предвидит цель движения, результат развития, отправление органа, то и сама по себе эта цель есть нечто предусмотренное; раз человек может себе представить по положению или направлению небесного мирового тела противоположное направление, даже бесчисленное множество других направлений, но при этом замечает, что если бы это направление отпало, то вместе с ним отпал бы целый ряд плодотворных, благоприятных следствий, почему в этом ряде он усматривает основание, из которого вытекает именно это, а не другое направление, --- значит, оно действительно изначала выбрано с изумительной мудростью, с учетом ее благодетельных следствий, из множества других направлений, которые, однако, существуют только в голове человека. Таким образом, для человека непосредственно, без всякого различия, принцип знания совпадает с принципом бытия, мыслимая вещь --- с действительной вещью, мысль о предмете --- с сущностью предмета, апостериорное --- с априорным. Человек мыслит о природе иначе, чем она есть; нет ничего удивительного, что в качестве основания и причины ее действительности он предполагает другое существо помимо нее самой, существо, данное лишь его голове, более того, --- представляющее собой сущность его собственного ума. Человек переворачивает естественный порядок вещей, он ставит мир вверх ногами в буквальном смысле этого слова вершину пирамиды он делает ее основанием, первое --- в уме и для ума, логическое <<почему>> он превращает в первое действительности, в обусловливающую причину. Основание вещи в уме предшествует самой вещи. Вот почему разумная, или рассудочная, сущность, мысленная сущность не только логически, но и физически составляет для человека изначальную сущность, основную сущность.



\section{}

Тайна телеологии --- в противоречии между необходимостью природы и произволом человека, между природой, какова она в действительности, и природой, как она представляется человеку. Если бы Земля занимала другое место, например место Меркурия, то все бы погибло от невероятной жары. Как мудро Земля оказалась водворенной именно туда, где ей подобает быть по ее составу. Но к чему сводится эта мудрость? Просто --- к противоречию, к противоположности по отношению к человеческой глупости, которая произвольно, мысленно ставит Землю не на то место, где она в действительности находится. Если ты с самого начала оторвешь друг от друга то, что существует в природе в неразрывном виде, как, например, астрономическое место мирового тела и его физический состав, то, разумеется, задним числом, единство природы представится тебе целесообразностью, необходимость --- планом, действительное, необходимое, совпадающее с своей сущностью местонахождение небесного тела покажется тебе разумным, правильно предусмотренным, правильно рассчитанным, мудро избранным местоположением, в противоположность тому неподходящему месту, которое ты измыслил и выбрал. <<Если бы снег был черного цвета или черный цвет был преобладающим в полярных странах\dots то все полярные области земли представляли бы собой мрачную пустыню, несовместимую с органической жизнью\dots Так распределение цветов в отношении тел\dots представляет одно из лучших доказательств целесообразного устройства мира>>. Разумеется, если бы человек не делал черное из белого, если бы человеческая глупость не распоряжалась произвольно природой, то и божественная мудрость не царила бы над землей.


\section{}

<<Кто внушил птице, что ей достаточно поднять свой хвост, чтобы полететь вниз, и что ей достаточно его опустить, чтобы взлететь выше. Нужно быть совершенно слепым, чтобы при полете птицы не заметить высшей мудрости, которая подумала за птиц>>. Разумеется, нужно быть слепым, но, умея распознавать природу, мы не понимаем человека, который из своей сущности делает прообраз природы, силу своего ума превращает в изначальную силу, полет птиц ставит в зависимость от понимания механики полета, отвлеченные от природы понятия превращает в законы, применяемые птицами в полете, --- наподобие всадника с его правилами верховой езды, наподобие пловца с его правилами плавания, --- однако с тем отличием, что применение искусства летания у птиц --- врожденное, изначала присущее. Но полет птиц вовсе не есть искусство. Искусство имеется лишь там, где есть то, что искусству противоположно, где орган выполняет функцию, не связанную с ним непосредственно и необходимо, не исчерпывающую его сущность, а являющуюся лишь особой функцией наряду с многими другими действительными или возможными функциями того же органа. Птица не может летать иначе, чем она летает, не может и не летать, она должна летать. Животное в состоянии делать лишь то единственное, что оно делает, а помимо этого не может ничего, и только потому оно может мастерски, непревзойденным образом осуществлять эту единственную деятельность, что ему недоступно ничто другое, что в этой одной функции оно исчерпывает всю свою способность, эта его функция совпадает с самим существом данного животного. Поэтому, если ты без предположения ума, действующего за животных, не можешь себе объяснить деятельность и функции животных, а именно низших животных, одаренных так называемым художественным инстинктом, то это происходит в связи с твоим предположением, будто предмет их деятельности есть такой же предмет, какими оказываются предметы твоего сознания и ума. Если ты себе представляешь продукты деятельности животных как художественные продукты, как результат произвола, то, разумеется, тебе естественно предположить и ум как их причину, --- ведь произведение искусства предполагает выбор, цель, ум, а, следовательно, поскольку тебя опять-таки опыт учит, что сами животные не мыслят, ты заставляешь другое существо мыслить за них. Вообще во всех умозаключениях от природы к бытию бога, посылка, предпосылка --- человеческого происхождения; неудивительно, что в результате получается человеческое или человекообразное существо. Если вселенная --- машина, то, естественно, должен быть и механик. Поскольку естественные существа так же друг к другу равнодушны, как человеческие индивидуумы, которых, например, может объединить и использовать для какой-нибудь произвольной государственной цели, например для военной службы, лишь высшая власть, постольку, разумеется, должен существовать регент, властитель, главнокомандующий природы, <<капитан облаков>>, если природа не захочет раствориться в <<анархии>>. Так, первоначально человек бессознательно превращает природу в человеческое творение, иными словами, превращает свою сущность в суть природы; поскольку же он после этого или одновременно замечает различие между произведениями природы и произведениями человеческого искусства, то эта его собственная сущность кажется ему чем-то другим, но аналогичным, сходным. Поэтому смысл всех доказательств бытия божия лишь логический или, скорее, антропологический, поколику и поелику и логические формы --- формы человеческого существа. <<Можете ли вы дать совет пауку, как протянуть нити от одного дерева к другому, от одного конька крыши до другого, от одной высоты по эту сторону реки к другой --- по ту сторону? ,,Ни в какой мере``; но думаешь ли ты, что тут нужен совет, что паук находится в том же положении, в каком бы оказался ты, если бы тебе пришлось головным путем решать эту задачу, что для него, как для тебя, существует ,,по ту сторону`` и ,,по эту сторону?``. Между пауком и предметом, к которому он прикрепляет нити своей паутины, такая же тесная связь, как между твоими костями и мускулами; ведь внешний предмет для паука есть не что иное, как опорная точка нити его жизни, опора для его орудия ловли. Паук не видит того, что видишь ты; все разделения, отличия, расстояния для него совсем не существуют, во всяком случае не существуют так, как они даны оку твоего разума. Что для тебя является неразрешимой теоретической проблемой, то делает паук, не применяя никакого ума и, следовательно, без всех тех затруднений, которые существуют только для твоего ума. <<Кто поведал травяным вшам, что они в большем изобилии найдут на ветке осенью себе пропитание в почке, чем в листе? Кто им показал путь к почке, к ветке? Для травяной вши, родившейся на листе, почка не только отдаленная, но совершенно неизвестная область. Я молюсь творцу травяной вши и червеца и умолкаю>>. Конечно, тебе приходится замолчать, если ты превращаешь травяную вошь и червеца в проповедников теизма, если ты им подсовываешь свои собственные мысли, ибо только для человекообразной травяной вши почка --- отдаленная и неизвестная область, но не для вши самой по себе; для нее лист существует не как лист и почка не как почка, но как усвояемое, как бы сродное ему химически вещество.

Поэтому только отражение твоего собственного глаза, заставляющее тебя в природе усматривать творчество всевидящего глаза, понуждает тебя выводить из головы мыслящего существа те нити, которые паук извлекает из своего заднего прохода. Для тебя природа только представление, только зрелище, ласкающее твой взор; поэтому ты веришь, ты думаешь, что то, чем восхищается твой глаз, движет и управляет также природой; небесный свет, в котором тебе является природа, ты превращаешь в небесное существо, ее создавшее, зрительный луч ты превращаешь в рычаг природы, зрительный нерв --- в двигательный нерв мирового целого. Выводить природу из мудрого творца --- значит родить детей одним взглядом, значит утолять голод аппетитным запахом пищи, значит благозвучием тонов двигать скалами. Если гренландец думает, что акула происходит из человеческой мочи, потому что для человеческого обоняния она пахнет мочой, то такая зоологическая гипотеза так же основательна, как космологическая гипотеза теиста, выводящего природу из ума по той причине, что она производит на человеческий ум впечатление чего-то разумного и целесообразного. Разумеется, явления природы представляются нам чем-то разумным, но причина этих явлений так же мало есть разум, как причина света есть свет.


\section{}

Почему в природе имеются уроды? Потому что у нее результат развития не дан вперед в виде цели. Почему имеются так называемые кошачьи головы? Потому что при образовании мозга природа не думает о черепе, не знает, что для покрытия его ей недостает костного вещества. Почему имеются лишние члены? Потому что природа не считает. Почему слева оказывается то, чему место справа, или справа --- чему место слева? Потому что природа не знает, что справа, что слева. Ссылки на уродства весьма обычны, они выдвигались уже старыми атеистами и даже теми теистами, которые освобождали природу из-под опеки теологии, --- они доказывают, что естественные образования --- непредвиденные, непреднамеренные, непроизвольные плоды; в самом деле, все основания, в том числе приводимые и новейшими натуралистами для объяснения уродов, будто они являются лишь следствием болезни зародыша, отпали бы, если бы с творческой или образующей силой природы одновременно были связаны воля, ум, предусмотрительность, сознание. Но хотя природа и не предвидит, она все-таки не оказывается слепой; хотя она и не живет в смысле человеческой, вообще субъективной, чувствующей жизни, но она и не мертва, и, хотя она и не творит согласно целям, все же ее плоды не случайны. Действительно, где человек определяет природу как мертвую и слепую, а ее произведения как случайные, там он свою сущность, притом субъективную, делает мерилом природы, там он ее определяет лишь по противоположности к самому себе, там он ее считает недостаточным существом, поскольку у нее нет того, что есть у него. Природа действует и творит повсюду, но в известной связи и по необходимости; эта связь обозначается человеком как ум, так как повсюду, где он видит связь, он находит смысл, материал для мысли, <<достаточное основание>>, систему. Необходимость природы не есть человеческая или логическая, метафизическая или математическая, вообще не абстрактная; ведь естественные существа --- не мысленные сущности, не логические или математические фигуры, но действительные, чувственные, индивидуальные существа; это необходимость чувственная, поэтому эксцентрическая, своеобразная, иррегулярная, представляющаяся человеческому воображению свободной или, по крайней мере, продуктом свободы благодаря такого рода аномалиям. Природа может быть понята только через самое природу; природа есть существо, <<чье понятие не зависит ни от какого другого существа>>: для нее одной имеет силу различие между тем, что есть вещь в себе, и тем, чем она является для нас; природа одна только является таким существом, к которому нельзя прилагать никакой человеческой мерки, хотя мы и сравниваем ее явления с аналогичными человеческими явлениями, применяем к ней, чтобы сделать ее понятной для нас, человеческие выражения и понятия, например: порядок, цель, закон, вынуждены применять к ней такие выражения по сути нашего языка, опирающегося лишь на субъективную видимость вещей.


\section{}

Религиозный энтузиазм перед божественной мудростью в природе есть лишь момент восторга; он направлен лишь на средства, но потухает в размышлениях о целях природы. Какое удивление вызывает сеть паука, как поражает муравьиная воронка в песке! Но для чего существуют эти мудрые приспособления? Для питания --- это цель, которую человек, как таковой, сводит к простому средству. Сократ говорил: <<Иные --- а эти иные --- животные и схожие с животными люди --- живут, чтобы есть, я же ем, чтобы жить>>. Как прекрасен цветок, его строение достойно удивления! Но на что это строение, для чего это великолепие? Только для украшения и защиты половых органов, которые человек, как таковой, прячет из стыда или даже увечит из религиозного усердия. <<Творец травяной вши и червеца>>, перед которым преклоняется и которым восхищается естествоиспытатель-теоретик, творец, усматривающий свою цель лишь в естественной жизни, не есть поэтому бог и зиждитель в религиозном смысле. Нет! Бог и зиждитель, как объект религии, есть лишь творец человека, и именно человека в его отличии от природы, человека, над природой возвышающегося, это творец, в котором человек осознает самого себя, которым олицетворяется его природа в отличие от свойств, коренящихся во внешней природе, и именно так, как он представляет себе эти свойства с точки зрения религии. Лютер утверждает: <<Вода, которая берется и проливается над ребенком при крещении, есть вода не творца, но бога-спасителя>>. Натуральная вода есть нечто общее и для меня, и для животных, и для растений, --- не такова крестильная вода. Первая связывает меня со всем остальным, вторая отличает меня от других существ природы. Не натуральная, а крестильная вода имеет религиозное значение, поэтому не творец и зиждитель природы, а творец крестильной воды оказывается объектом религии. Понятно, что творец естественной воды сам есть существо естественное, следовательно, не религиозное, не сверхъестественное. Вода есть сущность видимая, данная чувству, поэтому ее свойства и действия не приводят нас ни к какой сверхчувственной причине; крестильная же вода не есть предмет <<телесных очей>> --- это духовная, невидимая, сверхчувственная сущность, другими словами, наличная лишь для веры, действующая и существующая лишь в представлении, лишь в воображении; итак, это сущность, требующая в качестве своей причины сущность духовную, данную лишь в вере, в воображении. Натуральная вода очищает меня исключительно от моих телесных пятен и изъянов, крестильная же вода --- от моральных; первая утоляет лишь мою жажду в здешней, временной, преходящей жизни, вторая утоляет мое желание вечной жизни; первой свойственны лишь ограниченные, определенные, конечные действия, второй же --- бесконечные, всемогущие действия, выходящие за пределы природы воды, следовательно, действия, которые осуществляют и реализуют сущность божественного существа, не связанную ни с какими природными ограничениями, не связанную ни с какими границами опыта и разума, --- беспредельную сущность человеческой веры и способности воображения. Но разве творец крестильной воды не является и творцом воды натуральной? Как же он относится к творцу природы? Совершенно так же, как крестильная вода относится к воде натуральной; первой не будет, если не имеется второй, вторая есть условие, средство для первой. Так же творец природы только обусловливает творца человека. У кого нет под рукой натуральной воды, как смог бы он связать с ней сверхъестественные действия? Как может даровать вечную жизнь тот, кто не распоряжается жизнью временной? Как может воссоздать из праха мое тело тот, кто не имеет в своем распоряжении естественных элементов? Но кто же может быть повелителем и законодателем природы, кроме того, кто обладал могуществом и силой, чтобы из ничего создать природу простым актом своей воли? Поэтому объявляющий бессмысленным противоречием связь сверхъестественной сущности крещения с натуральной водой, пусть также назовет бессмысленной связь сверхъестественной сущности творца с природой; ибо между действиями крестильной воды и действиями воды натуральной такая же тесная или такая же слабая связь, как между сверхприродным творцом и столь естественной природой. Творец взят из того же источника, из которого выбивается сверхприродная, чудесная крестильная вода. В крестильной воде в наглядном примере твоему взору дано существо творца, существо божие. Как же ты можешь отвергать чудо крещения и другие чудеса, оставляя в силе сущность творца, т. е. сущность чуда? Другими словами, как отвергать мелкие чудеса, если ты признаешь великое чудо творения? Впрочем, в мире теологическом дело обстоит так же, как в мире политическом: мелких воров вешают, оставляя крупных на свободе.



\section{}

Провидение, обнаруживающееся в естественном порядке, в целесообразном устройстве и закономерности, не есть религиозное провидение. Религиозное провидение коренится в свободе, а первое --- в необходимости; религиозное провидение не ограничено и безусловно, первое же --- ограничено и зависит от тысячи условий; религиозное провидение --- особое, индивидуальное, первое распространяется лишь на целое, на род, предоставляя случаю единичное, индивидуальное. Один теистически настроенный естествоиспытатель говорит: <<Многие (многие? Все, для кого бог был чем-то большим, чем математической, фиктивной, отправной точкой природы) представляли себе сохранение мира, и в особенности человека, как нечто непосредственное, особое, --- словно бог управляет действиями всех тварей, руководя ими по своему благоусмотрению\dots Но такое особое вмешательство и контроль над действиями людей и других существ невозможно принять с точки зрения законов природы\dots В этом нас убеждает весьма малое попечение природы в отношении отдельных особей. Впрочем, природа так же мало ,,заботится`` о роде или виде. Вид сохраняется по естественным причинам, потому что вид есть не что иное, как совокупность индивидуумов, развивающихся и размножающихся через совокупление. Если отдельные особи и подвергаются случайным разрушительным воздействиям, то другие избегают этого. Множественность охраняет особей. Вместе с тем так же или, вернее, по той же причине, по которой умирают отдельные особи, вымирают и целые виды. Так вымер дронт и исполинский ирландский олень; так и теперь, вследствие преследования со стороны человека и все распространяющейся культуры, гибнут многие виды животных, вымирая в тех областях, где когда-то или недавно их было множество; таковы, например, тюлени на Южно-Шотландских островах; со временем они совсем исчезнут с лика земли. При богатстве природы они тысячами приносятся в жертву, беззаботно, без сожаления\dots Так же обстоит дело даже в отношении человека. И половина рода человеческого не доживает до двухлетнего возраста, причем эти дети умирают почти без сознания того, что они когда-либо жили. То же самое бросается в глаза и при несчастных случаях и мытарствах всех людей, как добрых, так и злых; все это не очень-то совмещается с особой поддержкой или содействием со стороны творца>>. Между тем это управление, это провидение, не имеющее специального назначения, не соответствует цели, сущности, понятию провидения; ведь провидение должно устранить случайность, однако если имеется лишь все общее провидение, то случайности сохраняют свою силу, поэтому такое провидение и не есть вовсе провидение. Так, например, одно из <<божьих установлении>> в природе, то есть результат естественных причин, сводится к тому, что соответственно числу лет и смертность людей выражается в определенных числах; так, например, на первом году умирает один ребенок из трех-четырех, на пятом году --- один из 25, на седьмом году --- один из 50, на десятом году --- один из 100; но то обстоятельство, что умирает именно этот ребенок, а другие три или четыре ребенка остаются жить, есть явление случайное, данным законом не обусловленное, зависящее от других, случайных причин. Так, <<брак есть божие установление>>, есть закон естественного провидения для размножения рода человеческого, следовательно, составляет для меня обязанность. Но обязанность эта мне ровно ничего не говорит о том, должен ли я жениться именно на этой, женщине, не является ли эта женщина негодной или бесплодной в результате случайного органического изъяна. Однако как раз при применении закона к определенному частному случаю, как раз в критический момент решения, в тисках нужды естественное провидение бросает меня на произвол судьбы, --- в действительности это естественное провидение и есть сама природа; в таком случае от нее я обращаюсь к высшей инстанции, сверхприродному провидению богов: их глаз обращается на меня как раз там, где кончается свет природы, их царство начинается как раз там, где конец царству естественного провидения. Боги знают, внушают мне, определяют то, что природой оставляется во мраке неизвестности, отдается во власть случая. Сфера случайного, как в обычном, так и в философском смысле, сфера <<положительного>>, индивидуального, непредусмотренного, не поддающегося исчислению, есть сфера богов, сфера религиозного провидения. А прорицания и молитвы представляют собой религиозные способы, при помощи которых человек превращает случайное, темное, неизвестное в предмет провидения, достоверности или хотя бы упования. Сравните по этому поводу высказывания Сократа (по Ксенофонту) о прорицаниях.



\section{}

Эпикур говорит, что боги обитают в промежутках между мирами. Прекрасно. Нас здесь, разумеется, не интересует подлинный смысл межмировых пространств Эпикура. Они существуют лишь в пустом пространстве, в бездне, зияющей между миром действительным и миром воображаемым, между законом и его применением, между действием и результатом действия, между настоящим и будущим. Боги --- воображаемые существа, мнимые, фантастические существа, которые, строго говоря, обязаны своим бытием не настоящему времени, а будущему и прошедшему. Боги, обязанные своим бытием последнему, являются тем, что уже больше не существует; это мертвецы, существа эти живут только в душе и представлении, их культ у некоторых народов и составляет всю религию, у большинства же --- значительную, существенную часть религии. Но бесконечно сильнее, чем прошлое, действует на душу будущее; прошлое оставляет лишь спокойное ощущение воспоминания, будущее же стоит перед нашим взором с ужасами ада или с небесным блаженством. Поэтому боги, восставшие из гроба, сами только тени богов; подлинные, живые боги, повелители дождя и солнца, молнии и грома, жизни и смерти, неба и преисподней, обязаны своим существованием лишь страху и надежде --- силам, распоряжающимся жизнью и смертью, расцвечивающим темную пучину будущего фантастическими существами. Настоящее в высшей степени прозаично, завершено, определенно; его нельзя изменить, восполнить, выделить. В настоящем представление совпадает с действительностью, богам в нем нет места, нет простора, настоящее --- безбожно. Будущее же есть царство поэзии, царство неограниченных возможностей и случайностей, будущее может быть тем или другим, оно может быть таким, каким я его желаю, или таким, каким я его страшусь. Оно еще не обречено суровой судьбе, не допускающей никаких измерений; оно еще витает между бытием и небытием в своих высотах над <<повседневной>> действительностью и явной наличностью; оно еще входит в сферу другого <<невидимого>> мира, мира, движимого не законами тяжести, а чувствительными нервами. Этот мир --- мир богов. Настоящее принадлежит мне, будущее --- богам. Сейчас я налицо; боги не могут у меня отнять настоящего мгновения, которое, впрочем, тотчас становится прошлым; даже божественное всемогущество не может случившееся превратить в не бывшее, как это утверждалось уже древними. Но буду ли я существовать в следующее мгновение? Зависит ли следующее мгновение моей жизни от моей воли или же оно находится в необходимой связи с настоящим мгновением? Нет? Тут бесконечное число случайностей; ежеминутно будущее мгновение может оказаться навеки отторгнутым от настоящего --- или неустойчивостью почвы под моими ногами, или падением крыши над моей головой, молнией, ружейным выстрелом, камнем, даже виноградиной, попавшей вместо пищевода в дыхательное горло. Но благодетельные боги не допускают этого резкого толчка, они своими эфирными, неуязвимыми телами заполняют поры нашего человеческого тела, открытые для всевозможных вредоносных воздействий, они минувшее мгновение связывают с наступающим; они служат посредниками между будущим и настоящим, они являются и владеют в непрерывной связи тем, чем люди --- эти пористые боги --- являются и владеют даже в промежутках, лишь с перерывами.

\section{}

Милосердие --- существенное свойство богов; но как могут они быть милосердными, если они не всемогущи, если они не свободны от законов естественного предвидения, то есть от цепей природной необходимости, если в индивидуальных случаях, в вопросах жизни и смерти, они не оказываются хозяевами природы и друзьями и благодетелями людей, следовательно, если они не творят чудес? Боги или, вернее, природа одарила человека телесными и душевными силами, чтобы он мог поддерживать свое существование. Но разве всегда достаточно этих естественных средств самосохранения? Не попадаю ли я часто в такое положение, в котором я безнадежно погиб бы, если бы божественная рука не задерживала беспощадного хода естественного порядка? Естественный порядок хорош, но всегда ли он хорош? Например, этот непрерывный дождь, эта продолжительная засуха вполне в порядке вещей, но если боги не помогут, не прекратят этой засухи, то в результате ее не погибну ли я, не погибнет ли моя семья, не погибнет ли даже целый народ? Даже христиане, подобно грекам, обращающимся к Зевсу, молят своего бога о дожде и думают, что он услышит их мольбу. В застольных беседах Лютера читаем: <<Была великая засуха, потому что долго не было дождя, и хлеб в поле уже стал сгорать; тогда доктор Мартин Лютер стал молиться и сказал, тяжело вздыхая: О! господи! Снизойди к нашей молитве ради твоего обетования\dots Я знаю, что мы от всего сердца к тебе взываем и вздыхаем с тоской, почему же ты нас не слышишь? В ближайшую же, ночь после того прошел обильный, благоприятный для растений дождь>>. Поэтому чудеса тесно связаны с божественным управлением и провидением, мало того, это --- единственное доказательство, раскрытие и проявление божеств, как сил и существ, отличных от природы; упразднить чудеса --- значит упразднить самих богов. Чем боги отличаются от людей? Только тем, что они не ограничены в том, в чем ограничены люди, что они всегда таковы, каковыми люди бывают только по временам, мгновениями. Правда, устранение ограничений сопровождается усилением и изменением свойств, но тождество остается в силе. Люди живут --- жизненность есть божественность, это --- существенное свойство, коренное условие божества, но --- увы!--- человеческая жизнь недолговечна, люди умирают, боги же бессмертны, обладая вечной жизнью; люди тоже счастливы, но не беспрерывно, подобно богам; люди также добры, но не всегда, и в этом, по Сократу, заключается отличие божества от человечества, что боги неизменно добры; люди также, согласно Аристотелю, наслаждаются божественным блаженством мышления, но у них духовная деятельность прерывается другими делами, другой деятельностью. Итак, у богов и у людей те же свойства, те же правила жизни, только у первых в их правилах нет ограничений, нет исключений, как у вторых. Потусторонняя жизнь есть не что иное, как продолжение настоящей жизни, не прерываемое смертью, так же точно божественное существо есть не что иное, как продолжение человеческого существа, не прерываемое природой, вообще --- непрерывное, неограниченное человеческое существо. Чем же отличаются чудеса от явлений природы? Как раз тем, чем божество отличается от человека. Действие или свойство природы, которое не оказывается благотворным в данном специальном случае, превращается посредством чуда в благотворное или, во всяком случае, в безвредное: благодаря чуду я не тону и не захлебываюсь в воде, когда по несчастной случайности падаю в воду; благодаря чуду огонь меня не сжигает, упавший на мою голову камень меня не убивает --- словом, чудо превращает существо, порою благодетельное, порою вредоносное, порою благосклонное к человеку, порою ему враждебное, в существо неизменно благое. Боги и чудеса своим существованием всецело обязаны исключениям из правил. Божество есть устранение человеческой ограниченности и изъянов, обусловливающих исключения из правил, чудо есть устранение изъянов и ограничений природы. Природные существа --- определенные и, следовательно, ограниченные. Эта их ограниченность в исключительных случаях обусловливает их гибельность для человека; но ограниченность эта с религиозной точки зрения не есть нечто необходимое, но произвольное, положенное богом, следовательно, устранимое, когда это требуется для надобности человека, то есть для его блага. Отвергать чудеса под тем предлогом, что они не соответствуют достоинству и мудрости бога, сообразно которым он изначала на вечные времена предустановил и предопределил все к лучшему, --- значит поступаться человеком ради природы, религией --- ради ума, значит во имя бога проповедовать атеизм. Если бог исполняет только те просьбы и желания человека, которые могут быть выполнены и без его помощи, осуществление которых не выходит за пределы границ и условий естественных причин, если бог доставляет помощь лишь при содействии искусства и природы и перестает ее доставлять, когда чудодейственная материя оказывается исчерпанной, то такой бог есть не что иное, как прикрытая именем бога олицетворенная необходимость.


\section{}

Вера в бога есть либо вера в природу (объективную сущность), как человеческое (субъективное) существо, либо вера в человеческое существо как сущность природы. Первая вера --- религия природы, политеизм, вторая --- духовно-человеческая религия, монотеизм. Обозначение политеизма как религии природы (безоговорочно и в общем смысле) имеет, лишь относительный смысл, смысл по противоположности. Политеизм жертвует собой природе, он одаряет природу человеческим оком и сердцем; монотеизм жертвует природой себе, человеческий глаз и сердце он наделяет силой и властью над природой; политеизм ставит человеческое существо в зависимость от природы, монотеизм --- природу от человеческого существа; первый утверждает: если нет природы, то нет и меня; второй утверждает обратное: если нет меня, то нет вселенной, нет природы. Отправной тезис религии таков: я --- ничто по сравнению с природой, по отношению ко мне все есть бог, все мне внушает чувство зависимости, все, хотя бы случайно, может мне доставить счастье и горе, спасение и гибель; поэтому все является предметом религии, --- первоначально человек не отличает причину от случайного повода. Религия, опирающаяся на это чувство зависимости без критической проверки, есть так называемый \underline{фетишизм}, основа политеизма. Что касается завершительного тезиса религии, то он таков: все ничто в сравнении со мной, весь блеск небесных созвездий, высших божеств политеизма меркнет перед величием человеческой души; все могущество вселенной --- ничто перед мощью человеческого сердца; вся необходимость мертвой, лишенной сознания природы --- ничто по сравнению с необходимостью человеческого, сознательного существа, ибо все для меня только средство. Но природа не была бы мне доступна, если бы она существовала самостоятельно, если бы она не исходила от бога. Если бы она существовала самостоятельно, следовательно, включала в себе основание своего бытия, то ее сущность была бы сущностью самостоятельной, самодовлеющей, не имеющей ко мне отношения, независимой от меня сущностью и бытием. Стало быть, значение природы как чего-то самого по себе ничтожного, как простого средства для человека восходит исключительно к акту творения; но это значение прежде всего раскрывается в тех случаях, когда человек впадает в нужду, подвергается смертельной опасности, оказывается в коллизии с природой, последняя же приносится в жертву благу человека, --- так происходят чудеса. Итак, предпосылка чуда --- творение, чудо есть заключение, следствие, истина творения. Творение так относится к чуду, как род или вид к отдельному индивидууму; чудо есть акт творения в виде особого, единичного случая. Другими словами: творение --- теория; практикой, применением является чудо. Бог --- причина, человек --- цель вселенной, то есть бог --- теоретическое первосущество, человек --- первосущество практическое. Природа для бога --- ничто, она --- простая игрушка его всемогущества, но только для того, чтобы в случае нужды, да и вообще природа не имела никакой силы над человеком. В творце человек освобождается от ограниченности своего существа, своей <<души>>; в чуде он освобождается от ограниченности своего существования, своего тела; в первом случае он превращает в существо природы свое невидимое, мыслящее и мыслимое существо, во втором случае --- свое видимое, практическое индивидуальное существо; в первом случае он узаконивает чудо, во втором --- он его реализует. Поэтому в чуде цель религии достигается чувственным, доступным путем--- власть человека над природой, божественность человека становится чувственно воспринимаемой истиной. Бог творит чудеса, но по просьбе человека и если не в ответ на жаркую молитву, то все же в интересах человека, в соответствии с его тайными, внутренними желаниями. Сарра засмеялась, когда господь обещал ей на старости лет сынишку, но и тогда еще, конечно, она больше всего думала и мечтала о наследнике. Итак, тайным чудотворцем является человек, но с течением времени --- а время раскрывает всякую тайну --- он становится и должен стать явным, видимым чудотворцем. Первоначально человек испытывает на себе чудеса, в заключение он их сам творит; первоначально он --- объект бога, в заключение он --- сам бог; первоначально бог находится только в сердце, в духе, в мыслях, в заключение он --- во плоти. Вместе с тем мысль скромна, чувственность нескромна; мысль молчалива и сдержанна, чувственность откровенно и без обиняков раскрывает себя, ее высказывания поэтому легко высмеять, если они находятся в противоречии с разумом, потому что здесь противоречие бросается в глаза, оно бесспорно. Вот почему современные рационалисты стыдятся верить в телесного бога, то есть в чувственное, наглядное чудо, но не стыдятся верить в бога нечувственного, то есть в нечувственное, скрытое чудо. Но придет время, когда исполнится пророчество Лихтенберга, когда вообще вера в бога, следовательно и в бога рационалистического, будет считаться таким же суеверием, каким в настоящее время считается вера в телесного, чудодейственного, то есть христианского, бога, когда вместо церковного света наивной веры и вместо сумеречного света рационалистической веры засияет, согревая человечество, яркий свет природы и разума.


\section{}

У кого для бога нет иного материала, помимо доставляемого естествознанием, житейской мудростью или вообще естественным взглядом на вещи, кто бога, следовательно, наполняет лишь естественнонаучными данными, под которыми подразумевается только причина или принцип астрономических, физических, геологических, минералогических, физиологических, зоологических и антропологических законов, тот должен быть настолько честен, чтобы воздержаться от имени божия, ибо принцип природы всегда есть естественная сущность, а не то, из чего слагается понятие бога. Безграничен произвол в употреблении слов. Но всего больше злоупотреблений и противоречивых истолкований встречается в отношении слов: бог и религия. Откуда этот произвол, откуда это смешение? Они происходят потому, что из страха или нежелания вступать в конфликт с мнениями, освященными давностью, старые названия сохраняются, но с ними связываются совсем другие понятия, возникшие с течением времени, ибо только название, только видимость управляет миром, даже миром религиозным. Так обстояло дело с греческими божествами, получившими с течением времени самые противоположные значения, так же обстоит дело с христианским богом. Религия сводится к атеизму, называющему себя теизмом; действительное христианство современности сводится к антихристианству, называющему себя христианством. Мир желает быть обманутым. Как церковь, превращенная в естественнонаучный кабинет, уже не дом божий и не должна так называться, так и бог, если его сущность и действия раскрываются лишь в астрономических, геологических, зоологических, антропологических делах, уже не есть бог; бог есть религиозный термин, религиозная сущность и объект, а не физическая, не астрономическая --- словом, не космическая сущность. В своих застольных беседах Лютер говорит: <<Deus et cultus sunt relativa: бог и богослужение взаимно связаны, одно не может существовать без другого, ибо бог непременно должен быть богом определенного человека или народа, он всегда находится в pracdicamento relationis, относится к чему-то другому. Бог хочет иметь ему поклоняющихся и его почитающих, ибо это одно и то же --- иметь бога и почитать его, это соотносительные понятия, как в браке муж и жена, --- одно не может быть без другого>>. Таким образом, бог предполагает людей, его почитающих и ему поклоняющихся; бог есть существо, чье понятие или представление зависит не от природы, но от человека, а именно от религиозного человека; объект поклонения имеется лишь, когда есть существо поклоняющееся, иными словами, бог есть объект, наличие которого дается лишь вместе с наличием религии, сущность которого дана лишь с сущностью религии; итак, бог есть то, что не существует вне религии, не существует как нечто отличное, независимое от нее; в боге объективно больше ничего не содержится, кроме того, что субъективно заключено в религии. Таким образом, существо, представляющее собой лишь философский принцип, следовательно, предмет философии, а не религии, не почитания, не молитвы, не чувства, существо, не исполняющее наших желаний, не выслушивающее наших молитв, есть только бог по имени, а не по своей сути. Звук есть предметная сущность, бог слуха; свет есть предметная сущность, бог зрения; звук существует только для уха, свет --- только для глаза; твое ухо обладает тем же, что имеется в звуке, --- это ударяемые, колеблющиеся тела, натянутые перепонки, студенистое вещество; в глазу же твоем заключены органы света. Превращать бога в физический, астрономический и зоологический объект или существо равносильно превращению звука в предмет зрения. Как звук существует только в ухе и для слуха, так и бог существует только в религии и для нее, только в вере и для веры. Как звук или тон в качестве предмета слуха выражает только сущность слуха, так бог, взятый только как предмет религии, веры, выражает только сущность религии, веры. Благодаря чему объект становится религиозным объектом? Как мы видели, только благодаря человеческой фантазии, способности воображения и человеческому сердцу. Все равно, молишься ли ты Иегове или Апису, грому или Христу, собственной тени, как это делает негр Золотого Берега, или своей душе, как это делает старый перс, молишься ли ты ветрам из живота или своему гению, --- словом, молишься ли ты чувственному или духовному существу, --- это безразлично. Предмет религии есть нечто, лишь поскольку это нечто является объектом фантазии и чувства, является объектом веры, именно поскольку предмет религии в качестве ее предмета не существует в действительности, а скорее находится с ней в противоречии, только постольку он есть объект веры. Так, например, человеческое бессмертие или человек как бессмертное существо есть объект религии, но именно поэтому это есть только предмет веры, --- ведь действительность как раз говорит о противоположном, о смертности человека. Верить --- значит воображать несуществующее существующим, значит, например, воображать, будто этот образ --- живое существо, будто этот хлеб --- мясо, будто это вино --- кровь, то есть предполагать, что есть то, чего нет. Если бы ты надеялся обнаружить бога при помощи телескопа на астрономическом небе или при помощи лупы в ботаническом саду, или при помощи минералогического молотка в геологических рудниках, или при помощи анатомического ножа и микроскопа во внутренностях животного или человека, то ты этим обнаружил бы полное непонимание религии. Ты найдешь бога только в вере, только в способности воображения, только в человеческом сердце, ибо он есть не что иное, как сущность фантазии или способности воображения, сущность человеческого сердца.


\section{}

<<Каково твое сердце, таков и твой бог>>. Каковы желания людей, таковы и их боги. Боги греков были ограниченными; это значит, их желания были ограниченны. Греки не желали жить вечно, они только не хотели стареть и умирать, но боялись они не того, что смерть неминуема, а боялись умереть вот сейчас, --- неприятное всегда приходит к человеку преждевременно; греки только не хотели умирать в цвете лет, хотели избегнуть насильственной, мучительной смерти; они стремились не к блаженству, они стремились лишь к счастью, они хотели только жить покойной, легкой жизнью; они не вздыхали, подобно христианам, по поводу того, что над ними тяготеют законы природы, потребности полового чувства, сна, еды и питья; в своих желаниях они не выходили из круга человеческой природы, они еще не были творцами из ничего, они еще не превращали воды в вино, они только очищали, дистиллировали натуральную воду и органическими способами превращали ее в божественный сок; содержание божественной, блаженной жизни они почерпали не из чистого воображения, но из элементов реального мира; небо богов они строили на основах этой земли. В раю христианской фантастики человек бы не мог умереть и не умер бы, если бы он не согрешил; у греков же человек умирал даже в счастливый век Кроноса, но умирал так сладко, словно засыпал. В этом представлении реализуется естественное человеческое желание. Человек не желает бессмертной жизни, он жаждет только продолжительной жизни, благополучной в телесном и духовном отношении, и хочет естественной, безболезненной смерти. Следовательно, чтобы отказаться от веры в бессмертие, нет необходимости прибегать к несвойственному человеку настроению отрешенности стоиков. Все сводится к тому, чтобы убедиться, что христианский символ веры основывается только на сверхъестественных, фантастических желаниях, и вернуться к простой, действительной человеческой природе. Божественное, то есть возможное, существо они не превращали в прообраз, цель и мерило существа действительного, но действительное существо делали мерилом существа возможного. Даже когда они очистили и одухотворили своих богов посредством философии, их желания остались на почве действительности, на почве человеческой природы. Боги --- реализованные желания, но высшее желание, высшее счастье философа, мыслителя, как такового, сводится к непрерывности мышления. Боги греческих философов --- по крайней мере греческого философа по преимуществу, философского Зевса, Аристотеля --- непрерывно мыслят; блаженство, божественность сводится к ничем не прерываемой деятельности мышления. Но эта деятельность, это блаженство само ведь протекает в пределах этого мира, в границах человеческой природы, правда, в здешних условиях --- с перерывами. Это есть действительное, определенное, особенное блаженство, поэтому с христианской точки зрения оно представляется ограниченным, жалким, противоречащим самой сущности блаженства: ибо бог христиан не ограниченный, а беспредельный, возвышающийся над всякой естественной необходимостью, сверхчеловеческий, внемировой, трансцендентный, а это значит: христиане во власти неограниченных, трансцендентных, выходящих за пределы мира, природы, человеческого существа, то есть абсолютно фантастических, желаний. Христиане хотят быть бесконечно счастливее, чем боги Олимпа; их желание --- небо, где исчезают все границы, вся природная необходимость, где исполняются все желания; на этом небе нет ни потребностей, ни страданий, ни ран, нет борьбы, нет страстей, нет препятствий, нет смены дня и ночи, света и тени, радости и горя, как все это есть на небе греков. Лютер, например, говорит: <<Где бог (именно на небе), там должны иметься все блага, каких только можно пожелать>>. То же самое говорится об обитателях рая в Коране по переводу Савари: <<Все их желания будут исполнены>>, но только их желания другого рода. Словом, предмет их веры не есть ограниченный, определенный бог, бог с известным именем --- Зевса, Посейдона или Гефеста, но абсолютный, безыменный бог, потому что предметом их желаний оказывается не какое-нибудь известное, конечное, земное счастье, не определенное наслаждение любовью или прекрасной музыкой, моральной свободой или мышлением, --- но наслаждение, охватывающее все возможные наслаждения, поэтому наслаждение исключительное, превосходящее всякое представление, всякое понятие, наслаждение бесконечным, безмерным, невыразимым, неописуемым блаженством. Блаженство и божество --- то же самое. Блаженство в качестве объекта веры, представления, вообще в качестве теоретического объекта, есть божество; божество как объект сердца, воли, желания, вообще как практический объект, есть блаженство. Впрочем, воля, как она понимается моралистами, не составляет специфической особенности религии, ибо я не нуждаюсь в богах, когда я могу чего-либо достигнуть собственной волей. Превратить мораль в существенную сторону религии --- значит сохранить название религии, отказавшись от ее сущности. Моральным можно быть и без бога, но нельзя быть без бога блаженным, блаженным в сверхъестественном, христианском смысле, ибо блаженство в этом смысле находится за пределами природы и человечества, не подчиняясь их власти, поэтому для своего осуществления это блаженство предполагает сверхъестественное существо, существо, которое может быть тем, что не под силу природе и человечеству, что выше всего человеческого. Поэтому, если Кант сделал из морали сущность религии, то он находился в таком же или приблизительно таком же отношении к христианской религии, в каком Аристотель стоял к греческой, поскольку последний теоретическую деятельность считал сущностью богов. Как бог, представляющий собой лишь спекулятивную сущность, лишь ум, еще не есть бог, так и моральное существо или <<олицетворенный моральный закон>> еще не есть бог. Конечно, и Зевс --- философ, когда он с улыбкой взирает с Олимпа на борьбу богов, но сверх того он есть нечто бесконечно большее; конечно, христианский бог есть и моральное существо, но сверх того нечто бесконечно большее; нравственность есть лишь условие блаженства. Истинная мысль, определяющая христианское блаженство, а именно в противоположность философскому язычеству, есть как раз та мысль, что только в удовлетворении всего человеческого существа можно обрести истинное блаженство, поэтому христианство позволяет быть причастным божеству, или, что то же, --- блаженству, --- телом, плотью. Впрочем, развитие этой мысли относится не к данной работе, а к <<Сущности христианства>>. Или, вернее, божество есть представление, истина и реальность которого сводятся к блаженству. Насколько простирается желание блаженства, настолько, не дальше, простирается представление о божестве. \emph{У кого больше нет сверхъестественных желаний, для того больше нет сверхъестественных существ}.
\end{document}
